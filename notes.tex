% Created 2018-12-12 Wed 11:46
% Intended LaTeX compiler: pdflatex
\documentclass[11pt]{article}
\usepackage[utf8]{inputenc}
\usepackage[T1]{fontenc}
\usepackage{graphicx}
\usepackage{grffile}
\usepackage{longtable}
\usepackage{wrapfig}
\usepackage{rotating}
\usepackage[normalem]{ulem}
\usepackage{amsmath}
\usepackage{textcomp}
\usepackage{amssymb}
\usepackage{capt-of}
\usepackage{hyperref}
\usepackage[left=2cm, right=2cm, bottom=2cm, top=2cm]{geometry}
\usepackage{parskip}
\usepackage{mathrsfs}
\usepackage{amsmath}
\def\R{\mathbb{R}}
\def\Z{\mathbb{Z}}
\def\N{\mathbb{N}}
\def\inte{\operatorname{int}}
\def\pos{\operatorname{pos}}
\def\aff{\operatorname{aff}}
\def\cone{\operatorname{cone}}
\def\ver{\operatorname{ver}}
\def\relint{\operatorname{rel\ int}}
\def\conv{\operatorname{Conv}}
\usepackage[T1]{fontenc}
\author{Hari}
\date{\today}
\title{Discrete Geometry 1}
\hypersetup{
 pdfauthor={Hari},
 pdftitle={Discrete Geometry 1},
 pdfkeywords={},
 pdfsubject={},
 pdfcreator={Emacs 26.1 (Org mode 9.1.13)}, 
 pdflang={English}}
\begin{document}

\maketitle
\tableofcontents

\section{Lecture 1}
\label{sec:org9233f1f}
\section{Lecture 2 \textit{<2018-10-17 Wed>}}
\label{sec:orgdeec859}
\subsection{Definitions}
\label{sec:org6f0c428}
\subsubsection{Positive half}
\label{sec:orgbba4c26}
\subsubsection{Affine hyperplane}
\label{sec:org7bbd004}
\subsubsection{(Question) What is the space of all oriented hyperplanes of \(\R^n\)}
\label{sec:orgb723694}
\subsubsection{Definition of pos?}
\label{sec:org1fa30c5}
\subsection{Radon's theorem}
\label{sec:orgd72f2ad}
Three cases

\begin{itemize}
\item If the size of M is greater than or equal to n+2, there is a radon partiiton
\item If the size of \(M\) is greater than or equal to \(n+1\), \(0\) is an appex of
\(M\) and \(0\neq M\). or \(|M| \ge n+2\).

then there is a partition of \(M_1\) and \(M_2\) of \(M\) such that \(\pos M_1
     \cap pos M_2 \neq \emptyset\).
\item partition is unique if and only i f
\begin{itemize}
\item \(|M| = n+2\) and \(n+1\) points of \(m\) are affinely dependent.
\item \(|M| = n+1\) and no \(n\) vectors of \(M\) are linearly independent.
\end{itemize}
\end{itemize}
\subsection{Definitions}
\label{sec:org2a7819e}
\subsubsection{Radon partition}
\label{sec:org310846e}
A partition \(M_1, M_2\) such that \(\conv M_1 \cap conv M_2 \neq \emptyset\) is called a \textbf{Radon partition}.
\subsection{Proof}
\label{sec:orgf558034}
\subsubsection{Part 1}
\label{sec:orgdd63d90}
Size of \(M\) greater than \(n+2\). Take \(x_1, \cdots, x_{n+2}\) pairwise disjoint points from \$M.

Then \(x_1, \cdots x_{n+2}\) are affinely dependent, meaning that that you can
choose scalars, real such that \(\sum \lambda_i x_i = 0\).

We can assume that for some \(1 \le j \le n+1\) holds

\(\lambda_1 >0, \cdots, \lambda_i >0\) and \(\lambda_{i+1} \le 0, \cdots, \lambda{n+2} \le 0\).

\(\lambda = \lambda_1 + \cdots + \lambda_j = -(\lambda_{i+1} + \cdots + \lambda_{n+2}\).

\(X=\frac1\lambda (\lambda_1 x_1 + \cdots \lambda_j x_j)\) a convex
combination \(\in \conv\{x_1, \cdots, x_j\}\).

$$\frac{1}{\lambda}(\lambda_1x_1 + \cdots + \lambda_{n+2}x_{n+2}) = 0$$


$$X= \frac{1}{\lambda}(\lambda_1x_1 + \cdots + \lambda_{j}x_{j})  = -\frac{1}{\lambda}(\lambda_jx_j + \cdots + \lambda_{n+2}x_{n+2}) = 0$$

a convex combination.

Thus \(X\) is inside both the intersection.
\subsubsection{Part 2}
\label{sec:orga1d5514}

\(\pos M\) is a cone with appex \ldots{} 

\textbf{A drawing}

For every point \(x_i \in M\), there is \(\alpha_i > 0\), such that \(\alpha_ix_i
    \in H'\),,

blah blah.

Keyword: matroids.
\subsubsection{Part 3}
\label{sec:orge08f40b}
I didn't type.

(A part.) A partition is unique.

Let \(\vert M \vert = n+2\) and \(x_1, \cdots, x_{n+2}\) are affinely dependent implies
\(x_1, \cdots x_{n+1}\) (dropped one point) have to be on a affine hyperplane.
Then by Radon's theorem, there is a partition \(M_1'\) and \(M_2'\) such that
the convex hulls of these intersect. But then we can add the element
\(x_{n+2}\) to \(M_1'\) or \(M_2'\) to form two different partitions. Thus we have
two Radon partitions of \(M\) and thus this is a contradiction.

Case for \(|M| \ge n+3\). Let \(\tilde{M} \subset M\) and \(|\tilde{M}| = n+2\)
such that \(M-\tilde M \neq \emptyset\). Then \(\tilde M_1, \tilde M_2\) is a
partition of \(\tilde M\).

\(\conv \tilde M_1 \cap \conv \tilde M_2 \neq \emptyset\).

$$\tilde M_1 \cap (M-\tilde M), \tilde M_2$$

and $$\tilde M_2 \cap (M-\tilde M)$$ forms two different radon partitions and
again, we have a contradiction.
\subsection{Questions}
\label{sec:org0495de3}
\subsubsection{About affine maps}
\label{sec:org92ae170}
We have an arbitrary affine map between a simplex (n+1 dimensional) and
\(\R^n\). implies, there exist faces of the simpleces such that the faces do
not intersect, but the images of the faces will intersect. Apparently this
follows from the Radon's theorem. A different formulation of Radon's theorem

\(a\colon T_{n+1} \rightarrow \R^n\)

\textbf{Question}: Replace the affine map by a continuous map and is it still true? \footnote{Wikipedia article about Radon'n theorem says that this is true. Crazy.}

\textbf{Question}: How many points in \(M \subset \R^d\) you should have to generate
that for \(n\ge 2\), there is a partition \(M_1, \cdots, M_r\) of \(M\) such that
the intersection of the convex hulls of \(M_i\) are non-empty. 

\textbf{Question}: More points, minimal number of points?
\section{Lecture 3 \textit{<2018-10-23 Tue>}}
\label{sec:org1f14059}
\subsection{Review}
\label{sec:org8131dcd}
\subsubsection{Radon's theorem}
\label{sec:orgcda1eb4}
\begin{enumerate}
\item If \(M \subset \R^n\) and \(\vert M \vert \ge n+2\), then there exists a
partition \(M_1\) and \(M_2\) of \(M\) such that \(\conv M_1 \cap \conv M_2 \neq
       \emptyset\).
\item If \(M \subset \R^n\) and either \(\vert M \vert \ge n+2\) and \(0\neq M\) on
\(\vert M \vert \ge n+2\), then there is a partition of \(M_1\), \(M_2\) of \(M\)
such that \(\pos M_1 \cap \pos M_2 \neq \empty\)
\end{enumerate}
\subsection{Charatheodery's theorem}
\label{sec:orgb3a1137}
\begin{enumerate}
\item Let \(M\subset \R^n\), then \(\conv M\) is the set of all convex combinations
of at most \(n+1\) points from \(M\). \footnote{What if \(M\) is a disc?}
\item Let \(M\subset \R^n\). Then \(\pos M\) is the set of all positive combinations
of at most \(n\) points from \(M\).
\end{enumerate}
\subsubsection{Proof}
\label{sec:org662c605}
\(x\in \conv M \implies\) there exists \(\lambda_1, \cdots, \lambda_n\) and \(x_1,\cdots, x_n \in M\). 

\(x=\lambda_1x_1+\cdots + \lambda_rx_r\) and \(\sum \lambda_i = 1\), \(\lambda_i \ge 0\).

Let the presentation be such that \(r\) is minimal. (We can do this because we
are taking minimum over natural numbers.) Let us assume that \(M \ge n+2\),
then there exists an affine dependence \(\mu_1x_2 + \cdots \mu_rx_r = 0\),
\(\mu_1 + \cdots + \mu_r = 0\) and not all \(\mu_i\) 's are zero.

(Basically the idea is that we assume the minimality of \(r\) and if \(r \ge
    n+2\), then there is an affine dependence, and then use this to contradict
the  minimality of \(r\).)\footnote{Solve the exercise for Helly's theorem.}
\subsubsection{Lemma about compactness of convex hull of compact set}
\label{sec:org5c41519}
\(M^{n+1} \times \Delta \rightarrow M\). Here the space on the left is the set
of all \(n+1\) points of \(M\) and \(\Delta\) is a simplex.

Now, it follows from the fact that image of a compact set is compact.
\subsection{Nearest points map and supporting hyperplane}
\label{sec:orge7f0c51}
\subsubsection{Lemma}
\label{sec:org28660e9}
Let \(K\subset \R^n\) be closed and convex. Then for every \(x\in \R^n\), there
is unique point \(x^1 \in K\) such that $$\Vert x - x^1\Vert = \inf\Vert x -
    y\Vert = d(x, K)$$
\subsubsection{Proof}
\label{sec:org1a50c73}
We can find a sequence of points \((y_n)\) in \(M\) such that the distance from
\(x\) is less than \(1/n\). Now, the sequence is Cauchy. Since, \(\R^n\) is
complete, it has to converge, and since \(K\) is closed, we are done. \footnote{Interestingly, convexity of the set is not used here. It's probably only
needed for the uniqueness.}

Uniqueness: Given \(X\), if there are two points \(x'\) and \(x''\) such that the
distances from \(x\) from these two points are the same. In the plane \(x, x',
    x''\), the triangle \(\Delta x x' x''\) exists. But then a perpendicular to the
side \(x'x''\) would be smaller than the distance to \(x'\) or \(x''\). This is a
contradiction. Hence the points have to be unique. (Here the convexity of
the set is used.)
\subsubsection{Definition of nearest point map}
\label{sec:orgc6dc274}
Given \(K\subset \R^n\) be a closed convex set. Then \(p_k \mathbb \R^n
    \rightarrow K\) is the nearest points map. (This is defined using the last
lemma.)

If \(x\in K\), then \(p_k(x) = x\). \(p_K\) is surjective. Usually it is not
injective, if \(K = \R^n\), then it is injective. 
\subsection{Properties of nearest point map}
\label{sec:org314a65c}
\subsubsection{Definition (supporting hyperplane)}
\label{sec:org63341e1}
A hyperplane \(H\) is a supporting hyperplane if \(a\) closed convex set in \(\R^n\) if 
$$H \cap K \neq \emptyset \textup{ and } K \subset H^- \textup{ or } K \subset H^+$$

If we take a \(u \in S(\R^n), \alpha \in \R^n\), \(H^+ = \{x\in\R^n \vert \langle x, u \rangle \ge \alpha\}\)
\(H^- = \{x\in\R^n \vert \langle x, u \rangle \le \alpha\}\).

A picture that I didn't draw 

Notions:
\begin{enumerate}
\item Supporting half space
\item Outer normal
\item Inner normal
\end{enumerate}
\subsubsection{Aim}
\label{sec:org1587d8d}
We want to prove: Given a convex body and take a point in the boundary. I
want to prove that there is a supporting hyperplane (?)
\subsubsection{Lemma}
\label{sec:org56f7d39}
Let \(\varphi \neq K \subset \R^n\) be a closed convex set. If \(x \in \R^n\setminus K\),
then the hyperplane \(H = \{y \in \R^n \vert \langle y, u \rangle = 1\}\) is a
supporting hyperplane of \(K\) at \(x' = p_k(x)\) where \(u=\frac{x-x'}{\langle
    x', x - x'\rangle}\).

A diagram I didn't draw (A convex body, x is a point outside, \(x'\) is the
closest element, meaning that \(x'\) is on the hyperplane and we have a
direction vector \(x - x'\), we normalize this vector. (\footnote{I think I made a mistake in what I wrote here.})) \footnote{If \(p_K\) is what we already know, we know that every point in \(p_k(\R^n \
K)\) \(\subset\) M\$ has a supporting hyperplane. We'll try to figure out more about
this set.}
\subsubsection{Proof}
\label{sec:org45c2cca}
\(H\) is a hyperplane and \(x' \in H\), then \(\langle x - x', x - x'\rangle \ge
    0 \implies \langle x, x-x' \rangle > \langle x', x-x'\rangle \implies \langle
    x, (x-x')/(\langle x', x-x'\rangle) \implies x\in H^+\)

Now we assume that \(H\) is not a supporting hyperplane, which means that
there is as point \(y\) inside \(K\cap (H+\setminus H)\). Consider the
triangle \(\Delta x x' y\). Since \(x\) is perpendicular to the \(yx'\), the angle
\(yx_1 x\) is actute. We kinda want to prove that there is a point on the line
segment that would minimize the distance from \(x\). The argument is similar
to the argument for last theorem. (The perpendicular from \(x\) would give a
point on the segment \(x'y\) that would be the minimum.) \footnote{I think I made mistakes in framing at the beginning of the paragraph.}
\subsubsection{Lemma}
\label{sec:org2e6f843}
Let \(K\subset \R^n\) be a closed convex set and \(x\in \R^n \setminus K\).
For a point \(y\) on the half-line emanating from \(x'=p_k(x)\) and containing
\(x\) holds

\(y' = p_K(y)=p_K(x) = x'\)
\subsubsection{Proof}
\label{sec:org9f5df80}
Let \(y \in [x', x]\), assume that \(y' \neq x'\). We'll try to arrive at a contradiction.

\(\Vert x - x' \Vert = \Vert x - y \Vert + \Vert y - x'\Vert \ge \Vert x -
    y\Vert + \Vert y - y'\Vert\) (The second part follows from the fact that \(y'\)
is the point in \(K\) that is closest to \(y\).)

We apply the inequality of triangle we get that \(\Vert x - x' \Vert \ge
    \Vert x - y\Vert\). This is a contradiction.

We do something similar when \(x\) is an element in the line segment \([y,
    x']\). (Not exactly similar, but try to arrive at a contradiction from
drawing some triangles and what-not.)
\subsubsection{Lemma Busemann and Faller's lemma}
\label{sec:orga58b529}
The function \(p_K\) does not increase the distance, therefore it is Lipschitz
with constant \(1\) and is uniformly continuous. This means that \(\Vert
    p_k(x) - p_K(y) \Vert \le \Vert x - y\Vert\).
\subsubsection{Proof}
\label{sec:org09ef471}
We assume that \(x' = p_K(x) \neq y' = p_K(y)\). (We draw a diagram.)

We kinda use principles similar to the last two theorems. I skipped writing
the proof.
\section{Lecture 4 \textit{<2018-10-24 Wed>}}
\label{sec:org29ed0d4}
\subsection{Review}
\label{sec:orgb4b2d9f}
\subsubsection{Nearest point map}
\label{sec:orgd406172}
The definition of the nearest point map for a convex set.

Recall that we use completeness of Real numbers for the existence of the map. \footnote{Did I write this statement correctly?}
\subsubsection{Some properties}
\label{sec:orge4454fa}
The nearest point map is identity in \(K\). 

Every point \(y\) on the half line emanating from \(x'\) containing is in the
fiber of \(x'\) with respect to \(p_K\).

\(f_K\) is a Lipschitz function with constant \(1\) and is hence continuous. 

Supporting hyperplane \(H \colon H \cap K = \emptyset\), \(K\subset H^{-}\). 
\subsubsection{Lemma}
\label{sec:org097905a}
If \(x \in \R^n - K\), then \(H=\{y \colon \langle y, x-x'\rangle = \langle x',
    x-x'\rangle \}\) is a supporting hyperplane of \(K\) at \(x'\).

The lemma says that at every point outside of \(K\), we can find a supporting
hyperplane. What we need to prove is that at every point on the boundary we
can find a supporting hyperplane. 
\subsection{Theorem}
\label{sec:org40d1892}
Let \(K\subset \R^n\) (here \(K\) is not equal to \(\R^n\) be closed convex set.
Then \(K\) is equal to the intersection of all its supporting half-spaces. 
\subsubsection{Proof}
\label{sec:orgee6fa5b}
Because \(K\) is not \(\R^n\), we have a point in the difference. Then there is
at least one supporting hyperplane, and therefore a supporting half space.
Let \(K'\) be the intersection of all of it's supporting hyperplanes of \(K\).
It is clear that \(K\) is a subset of \(K'\). To prove the inclusion from the
other side:

Let \(k'\) be an element in \(K'\). Then there exists a supporting hyperplane
\(H\) at \(x'=f_K(x)\) such that \(K \subset H^{-}\) and \(x \in inf H^{+}\). Thus
\(H\) separates \(H\) and \(K\), and more importantly, \(x\) is not an element of
\(K'\).\footnote{We're interested in spaces that can be formed by finitely many
intersections of hyperplanes. These will be called Polyhedra. An non-example is
a disc.}
\subsection{Theorem}
\label{sec:org45c9b15}
Let \(K\subset \R^n\) a closed convex set and \(x\in \partial K\). Then there
exists a supporting hyperplane for \(K\) containing \(x\). 
\subsubsection{Proof}
\label{sec:orgf4033e5}
We define the boundary of \(K\) first. Let \(x\in \partial K \iff (\forall U
    \in x \textup{ and open }) U \cap K \neq \emptyset\) and \(U\cap K^{c} \neq
    \emptyset\) and \(x_0 \in K\).

If \(x_0\) is a point in the boundary of \(K\), then there is a sequence \(y_n
    \in \R^n\) such that \(x_0\) is the limit of \(y_n\).

For every point \(x_n = f_K(y_n)\), there is a supporting hyperplane \(H_n\) at
\(x_n\). Let \(s_n\) be a sequence of half lines emanating from \(x_n\)
perpendicular to \(H_n\). Let \(S\) be a sphere with center at \(x_0 \in H\) of
small radius. Then this half line will intersect \(S\) at one point. Notice
that \(y_n\) is also an element of \(S_n\), then 

\(x_0 = \lim f_K(y_n')\) and \(y'_{k_n}\) subsequence of \(y_n'\) converging in
\(S, y_{k_n}' \rightarrow y_0 \in S\) and \(x_0 = \lim f_k(y_n') = \lim
    f_K(y_{k_n}')\) and \(y_0 = lim y_k' \implies f_k(y_0) = \lim f_k(y_{k_n}')\)
and \(x_0 = f_k(y_0)\) and \(y_0 \neq x_0\). \footnote{I don't understand what's happening at the end}
\subsection{Faces and Normal Cone}
\label{sec:org8998dc0}
\subsubsection{Definition}
\label{sec:org7484eab}
Let \(K\subset \R^n\) be a closed convex set. A face \(F\) of \(K\) is a subset of
\(K\) is a subset of the form \(F = K \cap H\) where \(H\) is some supporting
hyperplane of \(K\). 

Such a face is called a proper face while \(\emptyset\) and \(K\) are also faces
but called non-proper. (A diagram with \(\emptyset\) and \(K\).)

Examples: Triangles (here faces are the edges.) For a disc, then the faces
are points on the boundary. For a cube, the faces are the faces of the cube.
\subsubsection{Lemma about convexity of face}
\label{sec:orgb08d82c}
Every proper face of \(K\) is a closed convex set.
\subsubsection{Dimension}
\label{sec:org7e75409}
If \(F\) is a face of \(K\) and \(m=\dim F\), (Let affine hull of \(K\) is \(\R^n\).)
\begin{enumerate}
\item then \(m=0\) we call \(F\) a vertex of \(K\)
\item If \(m=1\) we call \(F\) an edge of \(K\)
\item If \(m=n-1\) we call \(F\) a \textbf{facet} of \(K\)
\item If \(m=n-2\) we call \(F\) a \textbf{ridge} of \(K\).
\end{enumerate}
\subsection{Lemma}
\label{sec:org5f0bd62}
Let \(F_0\) be subset of \(F_1\) faces of \(K\), then \(F_0\) is a face of \(F_1\).
\subsubsection{Proof}
\label{sec:org23c7690}
\(F_0\) is a face of \(K \implies\), therefore \(F_0 = K \cap H\), where \(H\) is a
supporting hyperplane for \(K\). \(H\) supporting hyperplane for \(K\) and
therefore for \(F_1\).

\(F_1 \cap H \subset K \cap H \subset F_1 \cap H\)

\(F_0 = F_1 \cap H\). 
\subsubsection{Remark}
\label{sec:org8852691}
The converse of the lemma does not hold. \(F_2\) is a face of \(K\) and \(F_0\) is
a face of \(F_2\) implies \(F_2\) is a face of \(K\). The last statement is not
true. Notice that in the above proof we need both of them to be faces of
\(A\).

The picture: A square with a half-disc glued to the right. \(F_0\) be a vertex
on the right side and \(F_1\) be the edge of the square containing \(F_0\). \footnote{Apparently the statement would be true for polytopes, i.e., the converse
holds for polytopes. This is one of the reason we're interested in polytopes.}
\subsection{Lemma}
\label{sec:org29dd215}
Let \(F_1, \cdots, F_k\) be faces of \(K\), then \(F=F_1\cap \cdots \cap F_k\) if a
face of \(K\).
\subsubsection{Proof}
\label{sec:org3abe1c2}
\(F_i = K \cap H_i\), where \(H_i = \{y \vert \langle y, u_i \rangle = 0\}\).
(\(K \subset H_i^{-1}\) \footnote{We need the next statement for making this assertion.}) We can assume that \(0 \in F\) which is the
intersection of all of them \(0 \in F = F_1 \cap \cdots F_k\).

\(u=u_1+\cdots+u_k\) (we can assume without loss of generality that \(u \neq
    0\); this can be attained by scaling one or more \(u_i\).)

\(H=\{y\vert \langle y, u\rangle = 0\}\). will be a supporting hyperplane for
\(K\) and \(F = K \cap H\). \(K\subset H^-\), \(y\in K\), \(\langle u, u\rangle =
    \langle y, u_1\rangle + \cdots + \langle y, u_k\rangle \le 0 + \cdots + 0\). \footnote{This is why we assumed that \(0\) is in \(K\), otherwise we'll have to play
around.}

The last statement implies that \(y\)

\(F = K \cap H\).

\(y\in F = F_1 \cap \cdots \cap F_k = (K\cap H_1) \cap \cdots \cap (K\cap H_k)\)

\(\langle y, u_1 \rangle = 0, \cdots, \langle y, u_k\rangle = 0\).

\(\langle y, u_1 + \cdots + u_k \rangle = 0\).

\(\langle y, u \rangle = 0 \implies y \in H\). 

\(y \in F \cap H \subset K \subset H\). 

\(y \in K \cap H \implies y \in K\) and \(y\in H\). 

\(\langle y, u_i \rangle \le 0\) 

\(\langle y, u \rangle = 0 = \langle y, u_1 \rangle + \cdots + \langle y,
    u_k\rangle\). \footnote{Question: what about infinite intersection.}
\subsection{Lemma}
\label{sec:orgabddf60}
\begin{itemize}
\item Let \(F\) be a face of a closed convex set \(K\) and \(x, \tilde x\) be an element
\end{itemize}
of the relative interior of \(F\). Then any supporting hyperplane of \(K\)
containing \(x\) must contain \(\tilde x\).
\begin{itemize}
\item If \(F, F'\) are faces of \(K\) and \(\relint F \cap \relint F \neq \emptyset\),
ten \(F = F'\).
\end{itemize}
\subsubsection{Proof}
\label{sec:orge368c62}
\(H\) supporting for \(F\). I didn't write this. \footnote{This means that we can choose a supporting hyperplane by choosing a
point inside relative interior.}
\section{Lecture 5 \textit{<2018-10-30 Tue>}}
\label{sec:org80e1a55}
\subsection{Review}
\label{sec:org6a8db23}
He did a review of stuff. 

\begin{enumerate}
\item \(K\) closed convex set and \(H\) is a supporting hyperplane of \(K\). Meaning
that \(F = K \cap H\) is a face. \(\phi_1, K\) (improper) face.
\item \(F\) face of \(K\) \(\implies F\) closed and convex.
\item \(F_1 \subset F_1\) faces of \(K \implies F_0\) a face of \(F_1\). Whereas the
converse of the statement is not true. \footnote{Remember the example with a square and a half disc glued to the right side of the square?}
\end{enumerate}
\subsection{Definition (Normal cone)}
\label{sec:org283689a}
Let \(K\subset \R^n\) be a closed convex set and \(x\in K\). The \textbf{Normal cone} at
\(x\) is the set at \(x\) is the set $$N(x) = -x + p_K^{-1}(\{x\})$$

The normal cone at \(x\) always contains \(0\). We'll draw some examples.

\begin{enumerate}
\item \textbf{A closed convex interval in \(\R\)}. Take a point \(x\) inside the interval.
Then \(N(x) = 0\). This is because the set of all points such that the
closest point is \(x\) is just \(x\).
\item If we go at the boundary of the convex set, then the set of points that
are closest to the point is the point and the whole half line containing
the point. Now \(N(x)\) is \([0, \infty)\) after translation. We can make a
similar argument for the point on the other side of the boundary.
\item \textbf{An interval in the plane}: let's say \([1, 3]\) inside \(\R^2\). Now, for
\(2\), there is a perpendicular line that is closest to \(2\). Now, if we
translate it, we get a line perpendicular to \(0\). Whereas, for \(3\) and
\(0\), they would be two dimensional spaces (half spaces.) We can get one
form another by doing orthogonal complement.
\item \textbf{A triangle inside plane}. All the points inside would give us \(0\).
Whereas, for a point on one of the edge (other than vertex), \(N(x)\) would
be a line perpendicular to the edge. For an edge, it would be a
two-dimensional space.
\item \textbf{Remark}: Notice that for all these examples, we were able to partition
the entire space using \(N(x)\). (I think the partition thing we are talking
about is about \(p^{-1}_K\). \(N(x)\) would always contain \(0\).
\end{enumerate}
\subsection{Lemma}
\label{sec:orgdb12c02}
\(N(x)\) is a closed convex cone. It consists of \(0\) and all outer normals of
\(K\) in \(x\). If \(x \in \inte K\), then \(N(x) = \{0\}\). 
\subsubsection{Proof}
\label{sec:orgb216d2f}
\(\lambda \ge 0, u \in N(x) \implies \lambda x \in N(x)\)

\(u, v \in N(x) \implies u + v \in N(x)\)

Without loss of generality, we can assume that \(x=0\). 

\(u\in N(0) \implies u \in p^{-1}_K\) and a lemma gives us that \(\lambda v\in
    p^{-1}_K(0)\) implies that \(\lambda u \in N(0)\).

\(u, v \in N(0) \implies 0 = p_K(u) = p_K(v)\), \(H_u = \{x\vert \langle u, x
    \rangle = 0\}\)

\(K\subset H_u^{-1}\). Supporting hyperplane at \(0\) of \(K\). \(H_u = \{x \vert
    \langle v, x\rangle = 0\}\), and \(K\subset K_v^{-1}\).

\(x \in K, \langle u + v, x\rangle = \langle u, x\rangle + \langle v,
    x\rangle \le 0 + 0 \le 0\).

\(x \in H^{-1} \implies K \subset H^{-1}\).

\(H\) is a supporting hyperplane, then \(p_{K}(u+v) = 0\).

What we proved is that, if we take a point, the positive multiple is inside.
We also proved that if there are two points inside, then the sum of them is
also inside. So it's like a cone. What about closed?

\(N(x) = -x + p^{-1}(\{x\}\). Now, because \(\{x\}\), is closed and \(p_K\) is
continuous, then inverse image is closed. Because the translation is an
isometry, we are done.
\subsection{Definition (Dual cone)}
\label{sec:orgae66ef8}
Let \(G\) be a cone, then \(\sigma = \{u \vert \langle \sigma, u\rangle \ge 0 \}\)
is called the \textbf{dual cone}. 
\subsection{Lemma}
\label{sec:orgff59613}
If \(\sigma\) is a cone with appex \(0\), then \(N_\sigma(0) = -\sigma\).\footnote{He didn't prove this, but it's obvious.}
\subsection{Lemma}
\label{sec:org67f5b52}
Let \(F\) be a face of a closed convex set of \(K\) and \(x, \tilde{x} \in \relint
   F\), then \(N(x) = N(\tilde x)\).
\subsubsection{{\bfseries\sffamily TODO} Proof}
\label{sec:orgedd1e7a}
The idea is that if there are two points in the relative interior of a face,
then the supporting hyperplane for these points are the same. We look at all
the normals at \(x\) and \(\tilde x\). \footnote{I think I'm missing some stuff.}
\subsection{A random story}
\label{sec:org3581585}
\(P \rightarrow \{F \colon F \textup{ a face of P}\), for every face, we can
talk about \(N(F)\) instead of a point in the relative interior. These two
sets, we put inclusion as a relation, these are anti-isomorphic \footnote{I think it means, the inclusion becomes opposite in the other space.} These
have some group structure and later can be used to construct affine Toric
variety. \footnote{The construction is similar to the construction of a toric variety.}
\subsection{Definition}
\label{sec:org0f29b45}
If \(F\) is a face of a closed convex set \(K\) and \(x\in \relint F\), then \$N(x)
is denoted by \(N(F)\) and is called the cone of normals of \(K\) in \(F\). \footnote{Connection with analysis: For a smooth real valued function from \(\R\),
we have a unique normal, whereas for a non-smooth point, there are several
different normals (or supporting hyperplane.) We'll do something similar in our
course.}
\subsection{Theorem}
\label{sec:org9e1a986}
Let \(K\) be a convex body in \(\R^n\) and \(x(F)\) are of the relative interior
points in \(F \neq \emptyset\) or \(K\). Then \(\{\relint N(x(F)) \vert F \text{
   face of } K\} = \{ \relint N(F) \vert F\textup{ face of } K \}\) is a
partition of \(\R^n\).
\subsubsection{{\bfseries\sffamily TODO} Proof}
\label{sec:org97c3028}
Since \(K\) is bounded, there exists \(\alpha\) non-negative, such that \(K\) is a
subset of \(H^{-1}(u, \alpha)\) where $$H(u, \alpha) = \{x \vert \langle x, u
    \rangle =\alpha\}$$

Let's take the intersection \(\cap_{K \subset H^{-1}}(u, \alpha) H^{-1}
    H^{-1}(u, \alpha)\)

There was a nice diagram. 
\subsubsection{Random stuff}
\label{sec:org33965be}
\(\forall u \in \R^n - \{0\}\), there exists a face \(F\) of \(K\), \(u\in N(F)\)
and \(0 \in N(K)\).

\(x \in \relint F\), and \(u\) is an outer normal of \(x\), then \(u\in \relint
    N(F)\).\footnote{This may or may not be true. Wasn't discussed in the class.}

He did an example with tetrahedra.

\(u \in \relint N(F_1) \cap \relint N(F_2)\), \(u\ in \relint x(F_1) \cap
    \relint x(F_2)\). This means that if we take \(u\) and add a point \(x(F_1)\),
\(p_K(u + x(F_1)) = x(F_1)\). This means that \(p_K(u + x(F_2)) = x(F_2)\)

(I missed parts of this argument.) We used boundedness of the convex body.
If it is unbounded, the family of normal cones do not cover.
\subsection{Definition (Normal fan)}
\label{sec:org4a71c36}
The family \(N(F)\) of \(K\) is called the normal fan of \(K\)
\subsection{Support and distance functionn}
\label{sec:org357acb5}
\subsubsection{Definition (support function)}
\label{sec:org4becf7a}
Let \(K\) be a non-empty convex body. The function \(h_K\) that maps \(\R^n
    \rightarrow \R\), \(h_K(x) = \sup_{x\in K} \langle u, x\rangle\) is the
\textbf{support function} of \(K\).\footnote{We use the fact that every compact function has a supremum.}

We can also say it is the supremum over a fixed \(x_0\).

There was a diagram
\subsubsection{Question?}
\label{sec:org96a68a9}
Given a ball, what is the normal fan of the space?

If we take an interior point, then we have \(0\). So we should go to the
boundary. But each point has a supporting hyperplane. Which means, that the
normal fan is all the half lines. The normal fan is a sphere. The normal fan
can be ugly when we have smooth convex body. But for polytopes, it's much
nicer.
\section{Lecture 6 \textit{<2018-11-06 Tue>}}
\label{sec:orgb1b1e5c}
\subsection{Support and distance function}
\label{sec:orgf4b9b45}
\subsubsection{Definition}
\label{sec:orgfffa18f}
Let \(K \subset \R^n\) be a convex body. The support function of \(K\) is
\(h_K\colon \R^n \rightarrow \R\), \(h_K(u)= \sup_{x\in K} \langle u,
    x\rangle\)
\subsubsection{Lemma}
\label{sec:orga5d3b80}
\(h_{K+a}(u) = h_K(u) + \langle u, a \rangle\)
\subsubsection{Proof}
\label{sec:orga649e2a}
\(K+a = \{x + a\vert x \in K\}\)

\(h_{K+a}(u) = \sup_{x \in K} \langle u, x+a\rangle \sup_{x\in K} \langle u x \rangle + \langle u, a\rangle\)

\(h_[p, q](u) = ?\), where \([p, q]\in \R^1\).
\subsubsection{Lemma}
\label{sec:org995430e}
Let \(K\subset \R^n\) be a convex body and \(u+0\), \(x_0 \in \partial K\), there
is a supporting hyperplane at \(x_0\), \(H_{x_0} = \{x \vert \langle x,
    u\rangle = \langle x_0, u\rangle \}\) where \(u\in N(x_0)\)
\begin{verbatim}
----------------------------X---------------------------
                           / \
                         -/ x \
                        /      \        
                       /        \       
                      /          \-     
                    -/             \    
                   /                \   
                  /                  \  
                -/                    \ 
               /                       \
              /                         \
\end{verbatim}
\begin{enumerate}
\item The hyperplane \(H_{u} = \{\langle x, u \rangle = h_K(u)\}\)
\item Every support hyperplane of \(K\) has the above form.
\end{enumerate}
\subsubsection{Proof}
\label{sec:org82f996d}
\(h_K(u) = \sup\langle x, u\rangle = \langle x_0, u\rangle\) for some \(x_0 \in
    K\) (because of compactness \footnote{There is also the notion of sequential compactness. Compactness and
sequential compactness are not equivalent.})

\(x_0 \in H_{u} \cap K\). Let \(y\in K\), then \(\langle u, u \rangle \le \sup
    \langle x, u \rangle= h_K(u)\)

\(y \in H^{-1}(u)\), thus \(K \subset H^{-1}\).

\(H\) is a supporting hyperplane. Then it cuts \(K\), \(x_0 \in K \cap H\), \(H
    =\{x \vert \langle x, u\rangle = \langle x_0, u \rangle \}\) where \(u \in
    N(x_0)\). \(N \subset H^{-1}\)
\subsection{Convex function}
\label{sec:org0846f38}
Let \(f \colon \R^n \rightarrow \R\), we say that \(f\) is convex if for every
\(x, y \in \R^n\) and \(0 \le \lambda \le 1\) holds that \(f(\lambda x +
   (1-\lambda)y) \le \lambda f(x) + (1-\lambda)f(y)\). If \(L \subset \R^n\) is an
affine subspace, then \(f_L\) is also convex.
\subsubsection{Definition}
\label{sec:org793df0d}
The function \(f\colon \R^n \rightarrow \R\) is \textbf{positively homogeneous} if
for every \(x\in \R^n\) and \(\lambda \ge 0\) the following holds: \(f(\lambda x)
    = \lambda f(x)\)
\subsection{Lemma}
\label{sec:org8344d2a}
Let \(f\) be positively homogeneous. Then \(f\) is convex if and only if for all
\(x, y, \in \R^n\) the following holds:

\(f(x + y) \le f(x) + f(y)\)
\subsubsection{{\bfseries\sffamily TODO} Proof}
\label{sec:orgada4f3f}
Let \(f\) be convex
\$f(x)/2 + f(y)/2 \(\le\) f((x+y)/2) \(\le\) \$

\(f\) satisfies the property, to show that \(f\) is convex.

\(f(\lambda x + (1- \lambda)y) \le f(\lambda x) + f((1-\lambda)y) = \lambda
    f(x) + (1-\lambda) f(y)\).
\subsection{Lemma}
\label{sec:org1afab7a}
A function \(f\colon \R^n \rightarrow \R\) is convex if and only if for all
\(x_1, \cdots, x_n\) and for all \(\lambda_1, \cdots, \lambda_n\) non-negative
such that \(\lambda_0 + \cdots + \lambda_m = 1\) the following holds

\(f(\lambda_0 x_0 + \cdots \lambda_m) \le \lambda_0 f(x_0) + \cdots + \lambda_m
   f(x_m)\)
\subsubsection{Proof}
\label{sec:orgba1e1c7}
\(\Leftarrow\), we put \(x_1, \cdots, x_m = y\), it follows trivially now.

\(\implies\), induction on \(m\). Assume that the condition holds with \(n-1\) on
every affine subspace of \(\R^n\) of dimension \(n-1\), i.e., \(f(\lambda_1 x_0 +
    \cdots + \lambda_{n-1}x_{n-1}) \le \lambda_0 f(x_0) + \cdots + \lambda_{n-1}
    f(x_{n-1})\). 

If \(\lambda_0\) is zero, we are done.

\(f(\lambda_0 x_0 + \lambda_1 x_1+\cdots + \lambda_n x_m) = f(\lambda_0x_0 +
    \lambda(1-\lambda_0)(\frac{\lambda_1}{(1-\lambda_0)} + \cdots + ? x_m)) \le
    \lambda_0 f(x_0) + (1-\lambda_0) f(\frac{\lambda_1}{(1-\lambda_0)} x_1 +
    \cdots + \frac{\lambda_m}{(1-\lambda_0} f(x_m))\) We are done.
\subsection{Lemma (Convex functions are continuous)}
\label{sec:orgb7544dc}
Every convex function \(f\colon \R^n \rightarrow \R\) is continuous
\subsubsection{{\bfseries\sffamily TODO} Proof}
\label{sec:org117e59e}
\(x_0 \in \R^n\), \(T = \conv \{x_0, \cdots, x_{m+1}\}\) regular simplex, such that 

\begin{verbatim}

             /|
            / |-\
           /  |  \
          /   |   -\
         /    |     \
        /     |      -\
       /      |        \
      /       |         \
     /       -+-         -\
    /     --/   \---       \
   /    -/          \---    -\
  /  --/                \---  \
 /--/                       \---\
-/------------------------------\-
\end{verbatim}
\(\Vert x_0 - x_1 \Vert = \cdots = \Vert x_0 - x_{n+1}\Vert\)

\(u\in U_{\partial (x_0)} = \{x \vert x - x_0\Vert < \partial \} \in T\)

\(x\) belongs to one of the simplex \(\{x_0, x_1, \cdots, \tilde{x_i}, \cdots,
    x_{n+1}\}\), \(x\in \conv \{x_0, x_1, \cdots, x_m\}\) and \(x = \lambda_0 x_1 +
    \cdots + \lambda_m x_m\) and \(0\le \lambda_1, \cdots, \lambda_m < \delta <
    1\).

\(\vert f(x) - f(x_0) \vert \le \vert \lambda_0 f(x_0) + \cdots +
    \lambda_mf(x_m) - f(x_0)\vert = \vert x_1 \vert f(x_0) - f(x_0)\vert +
    \cdots + \lambda_n (f(x_n) -f(x_0)\vert\)

\(f(x) > 0\) assume \(\le \lambda \lambda_1 \vert f(x_1)- f(x_0) \vert +
    \cdots + \lambda_m \vert f(x_n) - f(x_0) \vert \le (\lambda_1 + \cdots
    \lambda_n) M \le n \delta M < n\delta (M+1)\)

We can call the last value as \(\varepsilon\) and for given \(\varepsilon > 0\),
we can choose \(\delta\). Hence the function is continuous.
\subsection{Lemma}
\label{sec:org3c65ce2}
\(f \colon \R^n \rightarrow \R\) is convex if and only if \(T^{+}(f) =
   \{(x, \zeta) \vert x \in \R^n, \zeta \in \R, f(x) \le \zeta \}\) is a closed
convex \footnote{\(f(x) = 1/x\) and \(f'(x) = - 1/x^2\), but the function is not always
increasing (but on connected components.)}
\subsubsection{Examples}
\label{sec:org158c176}
\(y = x^2\) is a convex function.
\subsubsection{{\bfseries\sffamily TODO} Proof}
\label{sec:org0f58711}
\(\implies f\) is convex

\((x, \zeta), (y, \eta) \in T^{+}(f) \implies \zeta \ge f(x), \eta \ge f(y)\),
\(\lambda(x, \zeta) + (1-\lambda)(y, \eta) = (\lambda x + (1-\lambda)y,
    \lambda \zeta + (1-\lambda) \eta)\), \(f(\lambda x + (1-\lambda)y) \le \lambda
    \zeta + (1-\lambda) \eta\), is an element in the inverse.

\(\Leftarrow\) \(T^{+}(f)\) is convex and closed set in \(\R^{n+1}\)

\(f(\lambda x + \cdots + (1-\lambda)y)\)

\((x, f(x)) \in T^{+}(f)\) and similarly \((y, f(y)) \in T^{+}(f)\) these
implies that \(\lambda(x, f(x)) + (1-\lambda)(y, f(y)) \in T^{+}(f)\) this is
equal to \((\lambda x + (1-\lambda y, \lambda f(x) + (1-\lambda) f(y)\), then
\(f(\lambda x + (1-\lambda) y) \le \lambda f(x) + (1-\lambda) f(y)\).
\subsection{Lemma}
\label{sec:org818c40e}
Let \(f\in \R^n \rightarrow \R\) be positively homogeneous. Then \(f\) is convex
if and only if \(T^{+}(f)\) is a convex closed cone.
\subsubsection{{\bfseries\sffamily TODO} Proof}
\label{sec:orge36e9ae}
\(f\) is convex.

I didn't write this.
\subsection{Remark}
\label{sec:org3f0fee7}
Domain should not be restricted? 
\section{Lecture 7 \textit{<2018-11-07 Wed>}}
\label{sec:org8415104}
\subsection{Support and distance function (Review)}
\label{sec:org92d4e22}
\subsubsection{Definition}
\label{sec:orgd28a652}
\(K \subset \R^n\) non-empty convex body, we define the supoort function
\(h_K \colon \R^n \rightarrow \R\), \(h_K(x) = \sup_{x\in K} \langle x, u \rangle\)
\subsubsection{Lemma}
\label{sec:org79a52b5}
\(H_K(u) = \{x \in \R^n \vert \langle x , u \rangle = h_K(u)\}\) is a support
hyperplane for \(K\).
\subsubsection{Definition}
\label{sec:orgd176858}
\(f\colon \R^n \rightarrow \R\) is convex if for all \(\lambda \in [0, 1]\), and
for all \(x, y, \in \R^n\), \(f(\lambda x + (1-\lambda)y) \le \lambda f(x) +
    (1-\lambda)f(y)\) positive homogeneous if \(\forall \lambda \ge 0\) and
\(\forall x \in \R^n\), \(f(\lambda x) = \lambda f(x)\)
\subsubsection{Lemma}
\label{sec:orgf58189d}
\(f\colon \R^n \rightarrow \R\) positive homogeneous

\(f\) is convex \(\iff\) \(\forall x, y, \in \R^n\), \(f(x+y) \le f(x) + f(y)\).
\subsubsection{Lemma}
\label{sec:orgf86e04b}
\begin{enumerate}
\item Every convex function is continuous
\item \(f\colon \R^n \rightarrow \R\) is convex \(\iff\), \(T^{-1}(f) = \{(x, \zeta)
       \vert \R^n \times \R \vert f(x) \le \zeta\}\) is convex

\(f\) is convex \(\iff\) \(T^{+}(f)\) is closed convex cone in \(\R^{n+1}\).
\end{enumerate}
\subsection{Lemma}
\label{sec:org18b5f82}
A support function is positive homogenous and convex.
\begin{enumerate}
\item Proof
\label{sec:org20a1e41}
\begin{enumerate}
\item \(h_k(\lambda u) = \sup \langle x, \lambda u\rangle = \sup_{x \in K} \lambda
        \langle x, u \rangle = \lambda \sup \langle x, u \rangle = \lambda h_K(u)\)
\item \(x \in K \colon \langle x, u\rangle \le \sup_{x\in K} \langle x,
        u\rangle = h_K(u)\). \(\langle x, v \rangle \le \sup_{x\in K} \langle x,
        v\rangle = h_K(v)\)

Now \(\langle x, u + v\rangle \le h_K(u) + h_K(v)\), \(\implies\) \(\sup_{x
        \in K}\langle x, u+v\rangle \le h_K(u) + h_K(v)\).

\(h_K(u+v) \le h_K(u) + h_K(v)\), so \(h_K\) is convex.
\end{enumerate}
\end{enumerate}
\subsection{Lemma (linearity of \(h_K\))}
\label{sec:org16e9418}
\(h_K\) is linear on all elements of \(\sum(K)\) (normal cone.)
\subsubsection{Proof}
\label{sec:org6aae685}
\(u \in N(x_0)\), then \(h_K(u)\), \(H_K(u) = \{x \vert \langle x, u \rangle =
    h_k(u)\}\)
\begin{verbatim}
   \                 /-
    -\             /-
      -\         /-
        -\     /-
          -\ /-
           /o-
         -/   \-
       -/       \-
     -/           \-
   -/               \-
 -/                   \-
/                       X
 --         K         /-
   \-                /
     \--           /-
        \-       /-
          \-    /
            \-/-
             /
\end{verbatim}
So \(h_K\vert_{N(x_0)}\) is the scalar product \(\langle \cdot, x_0\rangle\).
\subsection{Definition}
\label{sec:orgc712cb0}
Let \(K\) be a convex body in \(\R^n\) and \(0\) is an element of the interior of
\(K\). (So the dimension of the convex body is \(n\), i.e., the body is full
dimensional.)

\(d_K\colon \R^n \rightarrow \R\) is defined as follows: \(d_K(\lambda \bar{x})
   = \lambda\) for \(\bar{x}\) element of the boundary of \(K\). (When the body is
symmetric, then one can define a norm.) Here \(\lambda > 0\).
\subsection{Lemma}
\label{sec:org6d9d024}
Let \(K\) be an n-dimensional convex body in \(\R^n\), 

\begin{enumerate}
\item A line \(g\) intersecting the boundary of \(K\) in three different points is
contained in a support hyperplane of \(K\), in particular, \(G \cap \inte K\)
is empty.\footnote{If we put \(\relint\) instead of \(\inte\), it will not work. This is why we
use the assumption, that we are using a full dimensional convex body rather than
a random one.}
\item Any ray emanating from an interior point of \(K\) intersects \(\partial K\) at
exactly one point.

Here \(A, B, C\) are points in the interior of the boundary and \(B\) is
inside the interior of \([A, C]\).

\(H\) be a supporting hyperplane through \(B\), and assume that \(A, C \notin
      H\), also no other point of \(g\) is in \(H\). Thus \(H\) separates \(A\) and \(C\),
contradiction.

\begin{verbatim}
                 --
                |  \-
                /    \-
               /       \-       
              /          \-                                       /-/
            A|             \-  C                        C     /----/
------------o/---------------\---------------------------o/-----g/
            /                  \-                     /---      /
           /                     \-               /---        -/
          |                        \-         /---           /
          /                          \-   /---              /
         ---\                          ---                -/
             ------\                                     /
                    ------\                            -/
                           -----\                     /
                                 ------\             /
                                        ------\    -/
                                               ----
\end{verbatim}
A ray can be thought of as a \(1\) dimensional half space.

\([g_0, g_1] = g\cap K\), a convex body in \(\R^1\).
\end{enumerate}
\subsection{Lemma}
\label{sec:org0ff701f}
\(d_K\) is positive homogeneous and convex.\footnote{Positive homogenous is a function \(f(\alpha x) = \alpha^k f(x)\).}
\subsubsection{Proof}
\label{sec:org83bca4b}
It is clear that the function is positive homogenous.

\textbf{Convexity}: We are going to use one of the equivalences on convexity, i.e.,
 \(f(x +y ) \le f(x) + f(y)\).

\(x = \lambda \bar{x}\) and \(y = \mu \bar{y}\). If \(x\in K\), then \(d_K(x) \le
     1\).

\(x = \lambda \bar{x}\) and \(y = \mu \bar{y}\).

\(\frac{\lambda}{\lambda + \mu}\bar{x} + \frac{\mu}{\lambda + \mu} \bar{y}
     \in K \implies\) \(1 \ge d_K(\frac{\lambda}{\lambda + \mu}\bar{x} +
     \frac{\mu}{\lambda + \mu} \bar{y}) = d_K(\frac{\lambda}{\lambda +
     \mu}{x}/\mu + \frac{\mu}{\lambda + \mu} {y}/mu) =
     d_K(\frac1{\mu+\lambda}(x+y)) = \frac{1}{\lambda+\mu} d_K(x+y)\).

\(d_K(x+y) \le \lambda + \mu = d_k(x) + d_k(y)\).\footnote{We now have the triangle inequality. But for the norm, we should also
have the following property: \(\Vert \lambda x \Vert = \vert \lambda \vert \Vert
x \Vert\).}
\subsection{Definition (Centrally symmetric)}
\label{sec:orge04f0c4}
A convex body \(K\) in \(\R^n\) is said to be \textbf{centrally symmetric} if there
exists a point in \(K\) such that \(K = \rho(K)\) where \(\rho\) is the central
symmetry with respect to \(c\).

Defining \(\rho\)? With respect to \(c\), \(\rho(x) = 2c - x\).

Examples:
\begin{enumerate}
\item Point.
\item Disc
\item Cube
\item Octahedron. Cross polytope? In \(\R^n\) we have a standard basis \((e_1,
      \cdots, e_n)\). We take \(\{e_1, \cdots, e_n\}\). The convex hull \(\{e_1,
      \cdots, e_n, -e_1, \cdots, -e_n\}\) is a cross polytope. A three
dimensional cross polytope is an Octahedron.
\end{enumerate}
\subsection{Theorem}
\label{sec:org1d04cc1}
Let \(K\) be a centrally symmetric convex body with \(0\in \inte K\), as its
center of reflection. Then \(d_K\) defines a norm in \(\R^n\) satisfying for all
\(\lambda \in \R\) and \(x, y \in \R^n\)
\begin{enumerate}
\item \(\Vert x \Vert = 0 \iff x =0\)
\item \(\Vert \lambda x \Vert = \vert \lambda \vert \Vert x \Vert\)
\item \(\Vert x +y \Vert \le \Vert x \Vert + \Vert y\Vert\)
\end{enumerate}
\subsection{Example}
\label{sec:orga90ed8d}
\begin{enumerate}
\item \textbf{Maximum norm} in \(\R^2\): \(\Vert (x_1, x_2)\Vert =- \max \{\vert x_1\vert,
      \vert x_2 \vert \} = d_K(x_1, x_2)\). What is the convex body that induces
this norm? A square (with the usual orientation.)
\item \textbf{Manhattan norm}: \(\Vert(x_1, x_2)\Vert = \vert x_1 \vert + \vert x_2
      \vert\). The convex body for this one would be a square (actually a cross
polytope; the slanted square.)
\end{enumerate}
\subsection{Polar bodies}
\label{sec:org4d1ee28}
Let \(K\) be a convex body in \(\R^n\) with \(0 \in \inte K\). For \(u \neq 0\), let

$$H_u^{-1} = \{x \in \R^n \vert \langle x, u \rangle \le 1\} = \cap_{x\in
   \partial K} H^{-1}_u$$

In particular, \(H^{-1}_0= \R^m\). The polar body of \(K\) is \(K^{*} = \cap
   H^{-1}_u\).
\subsection{Theorem}
\label{sec:org4a333b4}
Let \(K\) be a convex body with \(0 \in \inte K\). Then \(K^{**} = K\)
\subsubsection{Proof}
\label{sec:orgf1697e8}
\(K^{*} = \cap H^{-1}_u = \cap_{u \in K} \{x \vert \langle x, u \le 1 \} =
    \{x \vert \forall x \in K, \langle x, u \rangle \le 1\}\)

\(K^{*} = \{x \vert \langle x, K \rangle \le 1\}\)

\(K^{**} = \{y \vert \langle y, K^{*} \rangle \le 1\}\)

\(y \in K \implies \forall x \in K^*, \langle x, y \rangle \le 1\)

\(\langle K, y^{*} \le 1 \implies \langle K^{*}, y\rangle \le 1 \implies y
    \in K^{**}\), \(K \subset K^{**}\).
\section{Lecture 8 \textit{<2018-11-13 Tue>}}
\label{sec:org085c956}
\subsection{Polar Body}
\label{sec:org9a02b7c}
Let \(K\) be a convex body and \(0\in \inte K \subset \R^n\). For every \(u\in
   \R^n - \{0\}\), we define \(H_u^{-} = \{x \vert \langle x, u \rangle \le 1\}\)

Remember that \(H_K(u) = \{x \vert \langle x, u \rangle = h_K(u)\}\) is a
support hyperplane in the direction \(u\) of the convex body.

The polar body \(K^{*}\) is the intersection \(\cap H^{-1}_u\), when \(u\) is
element of \(K\). This is the same as \(\cap_{x\in \partial K} H_u^{-1}\). The
idea is that for \(m>1\), it is easy to see that \(H^{-1}_{mu} \subset H^{-1}_u\)
\subsection{Theorem}
\label{sec:orgf2053cf}
Let \(K\) be a convex body in \(\R^n\) such that \(0 \in \inte K\). Then \(K^{**} =
   K\).
\subsubsection{Proof}
\label{sec:org7eef9f2}
We use the notation \(\{x \vert \langle x, K \rangle \le 1\}\) to denote
\(K^{*}\). Similarly \(\{x \vert \langle x, K^{*} \rangle \le 1 \}\) to denote
\(K^{**}\).
\begin{enumerate}
\item \(y\in K\), from definition of \(K^{*}\), it follows that \(\forall x \in
       K^{*} \vert \langle x, y \rangle \le 1 \implies \langle K^{*}, y\rangle
       \le 1/2\) implies that \(y \in K^{**}\).
\item \(x \in K^{**} - K\), \(x' = p_K(x)\), \(u = \frac{x - x'}{\langle x', x -
       x'\rangle}\), then \(H_u = \{x \vert \langle x, u \rangle \le 1 \}\), \(x \in
       H^{+}_u\) (we assume that \(x\) is a point in the interior.) and \(K \subset
       H^{-1}_u\). \(K \subset H_u^{-1} \implies \forall x \in K\), \(\langle x, u
       \rangle \le 1 \implies \langle K, u \rangle \le 1\). This implies that \(u
       \in K^{*}\).

\(x \in K^{**}\) and \(u\in K^{*}\). Now \(\langle x, u \rangle \le 1\), \(x\in
       H^{-1}_u\) contradiction.
\end{enumerate}
\subsection{Random stuff}
\label{sec:org8befd97}
If \(K\) is a convex body, it will turn out that the support function of the
convex body will be the distance function of the dual body and vice-versa.
\subsection{Theorem}
\label{sec:orgb0ea489}
Let \(K \subset \R^n\) be a convex body and \(0 \in \inte K\), then \(d_K =
   h_{K^{*}}\) and \(d_{K^{*}} = h_{K}\).
\subsection{Lemma}
\label{sec:org7ffc552}
Let \(K_1 \subset K_2\), then \(K_2^{*} \subset K_1^{*}\). The proof is not too hard.
\subsection{Lemma}
\label{sec:orgd8edbd0}
Let \(K \subset \R^n\) and \(0 \in \inte K\). Then \(H_u\) is a support hyperplane
for \(K^{*}\) if \(u \in \partial K\).
\subsubsection{Proof}
\label{sec:orge6a35b6}
We know that \(K^{*} = \cap_{x\in \partial K} H^{-1}_u\). We take \(0\in \inte
    K\), and in each direction \(u\), we have a unique intersection with the body.
In each direction, we have exactly one point on the boundary.

\(K^{*} = \cap H^{-1}_u\) convex body \(0 \in \inte K^{*}\).

The proof is kinda easy. The proof involved constructing a new convex body,
\(\tilde{K} = \conv \{ \beta u \vert u \in \partial K\}\). Apparently there is
a problem with this proof. \footnote{Apparently it's slightly different in the book. Which book?}
\subsection{Proof of duality of distance and \(h_K\)}
\label{sec:orge880e68}
Let \(u \in \partial K \implies d_K(u) = 1\)

But we just argued that on the boundary, it is a support hyperplane. Thus
\(H_u = H_K(u) = \{x \vert \langle x, u \rangle = h_{K^{*}}(u)\}\) and \(\{x \vert
   \langle x, u \rangle = 1\}\). Thus \(h_{K^{*}}(u) = 1\).
\subsection{Theorem}
\label{sec:orgfcf7525}
Let \(K \subset \R^n\) be a convex body with \(0 \in \inte K\).

\(K^{+} = \Gamma^{+}(d_K) \subset \R^{n+1}\) and \(H = \{(x, 1) \vert x \in
   \R^{n}\} \subset \R^{n+1}\). Then
\begin{enumerate}
\item \(\partial K_{+}\) is the graph of \(d_K\) in \(\R^{n+1}\).
\item \$K\(_{\text{+}}\) \(\cap\) H\$is a translation of \(K\).
\item \(K^{*}_+ \cap H\) is a translate of \(K^{*}\)
\item \(K_{+}, K^{*}_+\) are convex with appex \(O\) in \(\R^n\).
\end{enumerate}
\subsection{Theorem}
\label{sec:org26f930a}
Every positive homogenous and convex function \(h \colon \R^n \rightarrow \R\)
is a support function \(h = h_K\) of a unique convex body \(K\) whose dimension
possibly \(<n\). \footnote{We didn't prove this and the theorem before.}
\subsection{Radon's theorem}
\label{sec:org4c6b1c7}
Let \(X\) be a set of points in \(\R^n\), and \(\vert X \vert \ge n+2\), then there
is a partition of \(X\) into \(P\) and \(N\), such that a convex hull of \(P\) and
\(N\) intersect.
\subsubsection{Proof}
\label{sec:orgda2463b}
One can assume that \(\vert X \vert = n+2\) and \(X = \{x_1, \cdots, x_{n+2}\).
There is an affine dependence \(\lambda_1 x_1 + \cdots + \lambda n+2 x_{n+2}
    = 0\) and \(\lambda_1 + \cdots + \lambda_n = 0\) and not all \(\lambda_i\) are
zero.

So we can write it in terms of \(\sum_{i \in P} \lambda_i x_i = \sum_{i \in
    N} -\lambda_i x_i\).

Now \(A = \sum_{x_i \in P} \lambda_i = \sum_{x_i \in N} \lambda)i > 0\).

Now it is pretty easy to see that there is a point in intersection.
\subsubsection{Questions?}
\label{sec:org649ea17}
Why not two? How many points should we have to say something like, we can
partition into \(100\) sets, but their convex hulls intersect.
\subsection{Affine space}
\label{sec:org407f99d}
\((\mathscr{E}, E, \theta \colon \mathscr{E} \times \mathscr{E} \rightarrow
   E)\)

\begin{enumerate}
\item For all \(A \in \mathscr{E}\), \(\theta_A \colon \mathscr{E} \rightarrow E\)
is a bijection. \(B \mapsto \theta(A, B)\).
\item For all \(A, B, C \in \mathscr{E}\), \(\theta(A, B) + \theta(B, C) =
      \theta(A, C)\).
\end{enumerate}

Example: \((E,E, \theta(u, v) = v - u)\) This is an example with \(E = \R^n\)
that we work with.
\subsection{Random stuff}
\label{sec:org677f99b}
A map \((\mathscr{E}, E, \theta) \rightarrow (\mathscr{F}, F, \theta)\).
\(\theta(\varphi, f), \varphi \colon \mathscr{E}\rightarrow \mathscr{F}\).

And \(f\colon E \rightarrow F\), and for all \(A, B \in \mathscr{E}\) and
\(\theta(\varphi(A), \varphi(B)) = f(\theta(A, B))\).
\subsection{Radon's theorem rephrased}
\label{sec:org64e0289}
Let \(T_n+1\) be a simplex and \(a \colon T_{n+1}\rightarrow \R^n\) be an affine
map. Then there exists faces \(\sigma\) and \(\tau\) of the simplex such that
their \(a\) images in \(\R^n\) would intersect.

Why is this the same?

Why affine map? If we have an affine map, he the image of the simplex is the
convex hull of all points on vertices.

What about continuous map?
\subsection{Continuous variant of Radon's theorem (Topological Radon)}
\label{sec:org240426b}
The continuous invariant is also true.
\subsection{Helly's theorem}
\label{sec:orgccd5c28}
Let \(K_1, \cdots, K_n\) be a collection of convex sets in \(\R^d\) such that
every subcollection of \(d+1\) of them intersects \(\neq \emptyset\), then the
complete family intersects.
\subsubsection{Proof}
\label{sec:orgcc02044}
Induction on \(n\). If \(n \le d+1\), there is nothing to prove. Let \(n \ge
    d+2\), assume that it holds for \(n-1\), consider the following points \(x_i \in
    \cap_{1 \le j \le n, j \neq i} K_j\). By induction hypothesis, this has to
intersect. In this way, have points \(x_1, \cdots, x_n\). By assumption, \(n\ge
    d+2\), we can apply Radon's theorem. And the point in the Radon's theorem
belongs to the intersection of everything.
\subsection{Helly's theorem*}
\label{sec:org26c90c9}
Let \(\{K_i \vert i \in I\}\) be a family of \textbf{convex bodies} in \(\R^n\) such
that every family of \(d+1\) of them intersects, then the whole intersection is
non-empty. The difference between these and the before theorem is that the
index set may be infinite. This follows from the above theorem because of
compactness. (Use the closed set intersection of compactness; the
finite-intersection property.)
\section{Lecture 9 \textit{<2018-11-14 Wed>}}
\label{sec:org453c054}
\subsection{Helly's theorem}
\label{sec:orgd9b5fbf}
Let \(K_1, \cdots, K_n\) be a collection of convex sets in \(\R^d\) such that
every subcollection of \(d+1\) of them intersects \(\neq \emptyset\), then the
complete family intersects.
\subsubsection{Is convexity important?}
\label{sec:orgfe6609a}
If we drop convexity of one set, we would not have such a result. Convexity
is an important assumption. It's not too hard to construct a
counter-example.
\subsubsection{Remark about infinite families}
\label{sec:org99e0881}
We can replace the finite family with an infinite family, but of convex
bodies.
\subsection{Charatheodery's theorem}
\label{sec:org8ca6a08}
If \(x\in \conv X\), then there exists a collection of at most \(d+1\) points in
\(X'\) of \(X\) such that \(x\in \conv X' \subset \conv X\).
\subsection{Colorful Charatheodery's theorem (theorem by Inere Bardney)}
\label{sec:orgda8d7ab}
Let \(S_1, \cdots, S_{d+1}\) be collections of points in \(\R^d\) and \(x \in
   \conv S_1 \cap \conv S_2 \cap \cdots \cap S_{d+1}\). Then there are points
\(x_1 \in S_1, x_2 \in S_2, \cdots, x_{d+1} \in S_{d+1}\) such that \(x\) is an
element of \(\conv \{x_1, \cdots, x_{d+1}\}\).\footnote{Think about \(S_i\) as having a unique color. I guess each \$S\(_{\text{i}}\)\$s are
disjoint.}
\subsubsection{Proof}
\label{sec:orgc9195ca}
A proof using infinite descent. 

Without loss of generality, we can assume that each of \(S_i\) are finite.
Without loss of generality, we can assume that \(x = 0\), \(d(x, \conv\{x_1,
    \cdots, x_{d+1}\})\) for any choice \(x_1 \in S_1, \cdots, x_{d+1} \in
    S_{d+1}\}\). This is a finite set.Let \(d\) be the minimum distance. If \(d=0\),
we are done. Since we have a set of points \(x_i \in S_i\), such that the
distance of the point \(x\) and the convex hulls formed by the points is zero,
which means \(x\) is in the convex hull and we are done.

Assume that \(d \neq 0\), then one of \(d\) minimizes, then there exit
\(\bar{x_1}\in S_1, \cdots, \bar{x}_{d+1} \in S_{d+1}\). But then, since the
convex hull is closed, the distance is attained at a point. There is a point
\(z\) such that \(d(x, z) = d\).

Let \(H\) be the hyperplane perpendicular or orthogonal to the vector \(z-x\),
oriented such that \(x \in H^{-1}\).

We prove first that \(S = \conv\{\bar{x}_1, \cdots, \bar{x}_{d+1}\} \subset
    H^{+}\) Assume that \(y \in S\) is in the interior of \(H^{-1}\). Consider the
triangle \(\Delta xyz\).We know that the angle \(xzy < \frac{\pi}{2}\). Second
thing we know is that the edge \(zy\) \(\subset\) S\$. \(xz \le xy\), because \(xz\) is
the minimum distance.

Consider a point \(t \in (z, y) \subset S\) such that the angle \(\angle xtz >
    \angle xzt\). But then \(\vert xz \vert > \vert xt\vert\). This is a
contradiction, since \(t\) is in \(S\) and \(xt\) minimizes the distance of \(x\) to
\(S\).

Thus there is no point of \(S\) in the interior of \(H^{-}\).

\(z \in \conv\{\bar{x}_1, \cdots, \bar{x}_{d+1}\} \cap H\). We can assume that
all of them are in \(H\), if not we can put zero. (Why?)

\(z \in \cone\{\bar{x}_1, \cdots, \hat{x}_i, \cdots x_{d+1}\}\)

Since \(x\in \conv S_j\), there is \(\tilde{x_j} \in S_j \cap H\) 

\(z \in \conv\{\bar{x}_1, \cdots, \tilde{x_j}, \cdots, \bar{x}_{d+1}\}\). We
call the last set \(\bar{S}\).

Consider the triangle \(\Delta xz\bar{x}_j\) and we know.
\begin{enumerate}
\item \([z, \bar{x}_j] \subset \bar{S}\)
\item \(\angle \tilde{x_j}zx < \frac{\pi}{2}\).
\item \(\vert xz\vert < \bar x \tilde{x}_j\vert\).
\end{enumerate}

This is similar to the last part, we have a contradiction. Taking a point \(t
    \in xz\) close to \(z\) such that \(\angle tzx < \angle ztx\) we get that \(\vert
    xt \vert < \vert xz\vert = d\). This is a contradiction since \(t \in \bar{S}
    = \conv\{\bar{x}_1, \cdots, \bar{x}_{d+1}\}\). We are done.
\subsection{Tverberg's theorem}
\label{sec:org5c170d1}
\href{https://en.wikipedia.org/wiki/Tverberg\%27s\_theorem}{Wikipedia} Let \(N=(d+1)(r-1)\) and \(X \subset \R^d\) with \(\vert X \vert \ge
   N+1\). Then there exists a partition \(X_1, \cdots, X_r\) of \(X\) such that

$$\conv X_1 \cap \conv X_2 \cdots \cap \conv X_r \neq \emptyset$$

Can we drop one point and still get the result? No, the number of points is
minimal. I didn't write the counter example.

A simplex. Consider a simplex \(\Delta_N = \conv \{x_0, \cdots, x_N\} \mapsto
   \R^n\) (this is an affine map)

\(X = \vert_{i=0}^n \lambda_i x_i \mapsto \sum_{i=0}^N \lambda_i f(x_i)\).

The affine map sends a face of the simplex to the convex hull of the
simplex.
\subsubsection{Question (Topological Tverberg Conjecture)}
\label{sec:org8a86fa6}
For all \(d \ge 1\), \(r \ge 2\) and \(n=(d+1)(r-1)\) and any continuous map
\(f\colon \Delta_N \rightarrow \R^d\), there exists \(G_1, \cdots, G_r\) pairwise
disjoint faces of \(\Delta_N\) such that \(f(\sigma_1) \cap \cdots \cap
    f(\sigma_r) \neq \emptyset\).

This holds if and only if \(r\) is a power of a prime. This has something to
do with the elementary Abelian groups and their structure. Elementary
Abelian groups are direct sums of cyclic groups \(\Z_p \oplus \cdots \oplus
    \Z_p\).
\subsection{Weak Tverberg theorem}
\label{sec:orgd156df1}
Let \(n\ge (r-1)(d+1)^2 + 1\) and \(X = \{x_1, \cdots, x_n\} \subset \R^d\).
There is a partition \(I_1, \cdots, I_r\) of \([n] = \{1, 2, \cdots, n\}\) such
that the intersection of the convex hulls 

$$\cap_{n \le j \le r} \conv\{x_i \vert i \in I_j\} \neq \emptyset$$

It's the same as last theorem, except we have more points.
\subsubsection{Proof}
\label{sec:org7622534}
\(k=(d+1)(r-1)\), \(s=n-k \ge (r-1)(d+1)^2 + 1 - (r-1)(d+1) = (r-1)(d+1)d +1\)

Now \(Y_1, \cdots, Y_{d+1} \subset X\) and each \(\vert Y_i\vert = s\), we claim
that the intersection is non-empty.

Proof: \(\vert Y_1 \cap Y_2\vert = \vert Y_1\vert + vert Y_2\vert - \vert Y_1
    \cup Y_2 \vert \ge n-k+n-k-n = n-2k \ge (r-1)(d+1)^2 - (r-1)\cdot 2 \cdot
    (d+1)\)

\(\vert Y_1 \cap Y_2 \cap Y_3 \vert = \vert Y_1 \cap Y_2 \vert + \vert
    Y_3\vert - \vert (Y_1 \cap Y_2) \cup Y_2\vert \ge n-2k+n-k-n = n-3k\)

\(\vert Y_1 \cap \cdots \cap Y_{d+1} \vert \ge n - (d+1)k \ge 1\).
\subsubsection{{\bfseries\sffamily TODO} Corollary}
\label{sec:orgaf65177}
If \(Y_1, \cdots, Y_{d+1} \subset X\) with \(\vert Y_i\vert = s\), then \(\conv
    Y_1 \cap \cdots \cap \conv Y_{d+1} \neq \emptyset\).

\(Y=\{\conv Y\vert Y \subset X, \vert Y\vert \ge s\}\), finite family of
convex sets.

Every subfamily of \(d+1\) elements non-trivially intersect. Then by Helly's
theorem there is

\(z \in \cap Y\). \(z\in \conv X\). But now Charatheodery tells that \(\exists x_1
    \subset X\) such that \(\vert X_1 \vert = d+1\) and \(z \in \conv X_1\).

\(\vert X - X_1 \vert = n - (d+1) \ge (n-1)(d+1)^2 + 1 - (d+1) \ge s\). \(z \in
    \conv(X - X_2) \implies \exists x_2 \subset X - X_1), x_2 = d+1\) and \(z =
    \conv X_2\). I can do this \(r\) times and we are done.
\subsection{Optimal colored Tverberg theorem}
\label{sec:orga918090}
\subsubsection{Topological Tverberg theorem}
\label{sec:org8b6b74e}
\(d \ge 1\), \(r \ge 2\), power of prime, \(f \colon \Delta_N \rightarrow \R^d\),
a continuous map. Here \(N=(d+1)(r-1)\).

Then there exists \(\sigma_1, \cdots, \sigma_r\) pairwise disjoint faces of
\(\Delta_N\), \(f(\sigma_1) \cap \cdots \cap f(\sigma_r) \neq \empty set\).
\subsubsection{Optimal colored version}
\label{sec:org6e1df15}
\(d \ge 1\), \(r\ge 2\) is prime. \(f\colon \Delta_N \rightarrow \R^d\) continuous
such that \(N=(d+1)(r-1)\) and \(\vert \Delta_N \vert= C_0 \sqcup \cdots \sqcup C_m\)
such that \(\vert C_i \vert \le r- 1\) implies that there exists \(\sigma_1,
    \cdots, \sigma_r\) (pairwise disjoint faces of \(\Delta_N\) such that
\(f(\sigma_1)\cap \cdots \cap f(\sigma_r) \neq \emptyset\).

For all \(i, j\), \(\vert \sigma_i \cap \sigma_j \vert \le 1\). About
Barany-Larwan conjecture for primes - 1.
\section{Lecture 10 \textit{<2018-11-20 Tue>}}
\label{sec:orga40348a}
\subsection{Lemma about convex functions}
\label{sec:org574cae4}
Let \(f\colon I \rightarrow \R\) be a convex function and \(x \le y \le z\) be
points in \(I_1\). Then the following two inequalities hold: 

$$\frac{f(y) - f(x)}{y-x} \le \frac{f(z)-f(x)}{z-x} \le \frac{f(z) -
      f(y)}{z-y}$$

One can think about this as the gradient/slope of lines that can be formed
using points on the function.
\subsubsection{Proof}
\label{sec:orgc91b62d}
\(x < y < z \implies\) there exists \(\lambda \in (0, 1)\), \(y = \lambda x + (1-\lambda)z\)

$$\frac{f(y) - f(x)}{y-x} = \frac{f(\lambda x + (1-\lambda)y) -
    f(x))}{\lambda x + (1-\lambda)z - x} \le \frac{\lambda f(x) + (1-\lambda)
    f(z) - f(x)}{(1-\lambda) z - (1-\lambda)x} = \frac{(1-\lambda)f(z) -
    (1-\lambda)f(x)}{(1-\lambda)z - (1-\lambda)x} = \frac{f(z) - f(x)}{z-x}$$

$$\frac{f(z) - f(y)}{z-y} = \frac{f(z) - f(\lambda x + (1-\lambda)
    y)}{1-\lambda x - (1-\lambda) z} \ge \frac{f(z) - \lambda f(x) -
    (1-\lambda)f(z)}{\lambda z - \lambda x}$$

The last function is equal to \(\frac{\lambda f(z) - \lambda f(x)}{\lambda
    z - \lambda x} = \frac{f(z) - f(x)}{z-x}\)
\subsection{Theorem (Jensen's inequality)}
\label{sec:org51585e7}
Let \(f\colon C \rightarrow \R\) be a convex function. Then for every \(x_1,
   \cdots, x_m \in C\) and \(\lambda_1, \cdots, \lambda_m \ge 0\) with \(\sum
   \lambda_i = 1\) holds

$$f(\lambda_1 x_1 + \cdots + \lambda_m x_m) \le \lambda_1 f(x_1) + \cdots +
   \lambda_m f(x_m)$$
\subsection{Definition}
\label{sec:orgbf6360d}
Let \(f\colon X \rightarrow \R\) where \(X \subset \R^n\). The function \(f\) is
Lipschitz on \(Y \subset X\) if there exists \(L > 0\), and \(x, y \in Y\), \(\vert
   f(x) - f(y) \vert \le L \Vert x - y \Vert\). The function \(f\colon X
   \rightarrow \R\) is locally Lipschitz on \(X\) if for all \(x\in X\), there exists
\(x\in U\), a neighbourhood of \(x\) in \(X\), \(f\vert_U\) is Lipschitz.
\subsection{Theorem}
\label{sec:org86192db}
Let \(f\colon C\rightarrow \R\) be a convex function. Then \(f\) is Lipschitz on
every compact subset of the interior of \(C\). In particular, \(f\) is continuous
on the \(\int C\).
\subsubsection{Proof}
\label{sec:orga284a58}
It suffices to prove that \(f\) is locally Lipschitz. This is true, because at
each point in the compact set, find a neighborhood where it is locally
Lipschitz. Now this is a cover, hence there is a finite subcover. Choose
Lipschitz constant to be the maximum of all the elements and we are done.

Let \(X\) be element of \(\int C\), let \(N(X, \varepsilon)\) be an open ball
around \(x\) of radius \(2\varepsilon\). Then there exists \(\varepsilon, \alpha >
    0\) such that \(N(2\varepsilon) \subset \int C\) and \(f\vert_{N(2\varepsilon)}\)
is bounded from above by \(\alpha\).

It suffices to prove that \(f\) is bounded above on a simplex containing \(x\)
in the interior. Let \(x \in \int\{\conv\{x_0, \cdots, x_{n}\}\) where \(x_0,
    \cdots, x_m \in \int C\) and affinely independent.

If \(y \in \Delta\), then \(y = \lambda_0 x_0 + \cdots + \lambda_m x_n\) where
\(\lambda_0, \cdots, \lambda_m \ge 0\), and \(\sum \lambda_i = 1\).

\(f(y) = f(\lambda_0 x_0 + \cdots + \lambda_m x_m) \le \lambda_0 f(x_0) +
    \cdots + \lambda_m f(x_n) \le \vert f(x_0) \vert + \cdots + \vert f(x_n)
    \vert = \alpha\).

There exists \(\gamma > 0\), such that the absolute value \(\vert
    f\vert_{N(2\varepsilon)}\vert\) is bounded by \(\gamma\) from above.

\(x = \frac{1}{2}(2x-y) + \frac{y}{2} \implies f(x) = f(\frac{1}{2}(2x-y) +
    \frac{1}{2}y) \le \frac{1}{2} f(2x + y) + \frac{1}{2} f(y)\)

\(2f(x) - f(2x + y) \le f(y)\) (missed some details following)

\(\Vert 2x - y - x\Vert = \Vert x - y \Vert < \varepsilon\)

\(\alpha \ge f(y) \ge 2 f(x) - f(2x - y) \ge 2 f(x) - \alpha\)

\(f\) is Lipschitz on \(N(\varepsilon)\)

\(\Vert w - x \Vert \le \Vert w - y\Vert + \Vert y - x\Vert \le \varepsilon +
    \varepsilon\). \(y, z \in N(\varepsilon)\), \(w\) is a point on the ray emanating
from \(z\) containing \(y\) such that the distance of \(w\) to \(y\) is exactly
\(\varepsilon\) (A diagram was drawn)

\(f\vert_{[w, z]}\) is a convex function on interval \([w, z]\).

$$\vert{f(y) - f(z)}{\Vert y -z\Vert} \le \frac{f(w) - f(z)}{\Vert w - y
    \Vert} \le \frac{2\gamma}{\varepsilon}$$

This implies that \(\vert f(y) - f(z)\vert \le \frac{2\gamma}{\varepsilon}
    \Vert y - z\Vert\)

We would be done if we had the opposite one, \(\vert f(y) - f(z)\vert \le
    \frac{2\gamma}{\varepsilon} \Vert y - z\Vert\). But this is true using
similar argument. Therefore the function is locally Lipschitz (this would
also imply continuity.)    
\subsection{Combinatorial theory of Polytopes and Polyhedral sets}
\label{sec:org24ba44f}
\subsubsection{A boundary complex of a polyhedral set}
\label{sec:org4c12e0d}
\begin{enumerate}
\item Theorem
\label{sec:org56436e9}
Every Polytope contains or has finitely many faces and all of them are
Polytopes.
\begin{enumerate}
\item Proof
\label{sec:org577e4ec}
Let \(P\) be a Polytope \(P = \conv\{x_1, \cdots, x_r\} \subset \R^n\) b ea
Polytope and \(F = P cap H\) be a face of \(P\), where \(H = \{x \vert \langle
      x, a \rangle = \alpha \}\) is a support hyperplane.

This means it intersects the Polytope and the whole Polytope is on one
side. Thus \(P \subset H^{-1}\).

We can assume that \(\langle a, x_1 \rangle = \cdots = \langle a,
      x_s\rangle = \alpha\) for \(1\le s \le r\).

\(\langle a, x_{s+1}\rangle = \alpha - \beta_{s+1}\) where \(\beta_{s+1} >
      0\), \(\langle a, x_r \rangle = \alpha - \beta_r\) where \(\beta_r > 0\).

let \(x\in P\) and therefore \(x = \lambda_1 x_1 + \cdots + \lambda_r x_r\)
for \(\lambda_i > 0\) and \(\sum \lambda_i = 1\).

\(\langle a, x\rangle = \langle a, \sum \lambda_i x_i \rangle = \sum
      \lambda_i \langle a, x_i \rangle = \sum_{i=1}^{r} \lambda_i \alpha -
      \sum_{i=s+1}^{r} \lambda_i \beta_i\)

\(\langle a, x \rangle = \alpha - \sum_{i=s+1}^{r} \lambda_i \beta_i\)

\(x \in F = P \cap H\), \(\langle a, x\rangle = \alpha - \sum_{i=s+1}^{r}
      \lambda_i \beta_i = \alpha\), \(\beta_i > 0\) implies that \(\lambda_{s+1}=
      \cdots = \lambda_r = 0\). \(X = \lambda_1 x_1 + \cdots + \lambda_s X_s\).

\(F = \conv \{x_1, \cdots, x_s\}\) a Polytope.
\end{enumerate}
\item Theorem (Krein-Milman theorem)
\label{sec:orgddc93a5}
Every convex Polytope is a convex hull of its vertices. \(P = \conv(\vert
     P)\). (Vertices are the zero-dimensional faces of a Polytope.)
\begin{enumerate}
\item Proof
\label{sec:orgd0ce790}
\(P = \conv\{x_1, \cdots, x_r\}\). \(\vert P \subset P\), \(\conv (\vert P)
      \subset P\).

We can assume that for every \(i \in \{1, \cdots n\}\), \(x_i \notin \conv
      \{x_1, \cdots, \hat{x}_i, \cdots, x_r\}\) We can do this and get a minimal
set of points.

Now we need to prove that every \(x_i\) is a vertex. 

\(y_i = p_{P_i}(x_i)\). There exists a supporting hyperplane \(H'_i\) to \(P'_i\)
orthogonal to \(x_i - y_i\) and containing \(y_i\) with the property that \(P_i
      \subset H_i'\) and \(x_i \in \int H_i^{+}\) (a diagram was drawn)

Let \(H_i'\) be a the parallel Hyperplane to \(H_i\) passing through \(x_i\)
(such a hyperplane is unique, because we are in Euclidean Geometry.)

\((H_i)^{-} \subset \int (H'_i)^{-}\) implies that \(x_1, \cdots, \hat{x}_i,
      \cdots, x_r \in H_i^{-1} \subset \int (H'_i)^{-}\) and \(H'_i \cap P =
      \{x_i\}\). Thus \(x_i\) is a vertex. We are done.
\end{enumerate}
\item Remark
\label{sec:orgb2e574f}
\(P = \conv \{x_1, \cdots, x_n\}\), we assume that these \(x_i\) are vertices.
\item Definition (Polyhedral set)
\label{sec:orgb580148}
A finite intersection of closed half spaces is called a Polyhedral set.
\item Theorem
\label{sec:org122a4da}
Every Polytope \(P\) is a bounded Polyhedral set.
\item Remark
\label{sec:org74dd4ba}
A Polytope is a convex hull of finitely many points.

Also a finite bounded intersection of closed half space. (finitely many
vectors and distances.)
\end{enumerate}
\section{Lecture 11 \textit{<2018-11-21 Wed>}}
\label{sec:org829803c}
\subsection{Definition}
\label{sec:org21646e5}
A polyhedral set is a finite intersection of closed half-spaces
\subsection{Theorem}
\label{sec:orgc76ceb2}
Every polytope is a bounded polyhedral set.
\subsubsection{Proof}
\label{sec:org399cfa1}
\(P\) is a polytope. The aim is to show that it is the intersection of
finitely many closed half spaces \(H_i^{-}\).

It suffices to prove that \(P\) is an intersection of finitely many closed
half spaces. This would mean that it would be a Polyhedral set. Boundedness
follows from the boundedness of \(P\).

\(P' = \cap_{H_i \textup{ suppporting hyperplane of a facet of P}} H_i^{-1}\)
where \(H_i\) is the supporting hyperplane of a facet of \(P\). We need to prove
that \(P' = P\).

A theorem implies that the intersection is finite. (Is this because of
compactness?)

Suppose the contrary. \(x_0 \in P' \setminus P\). The idea is to find a facet
that would separate \(P\) and \(x_0\), which would give us a contradiction. Let
\(A\) be the union of all affine subspaces spanned by \(\aff(x_0 \cup
    \textup{face of dim} \le\) where \(F\) is a face of \(P\) and \(\dim F \le n-2\).
This is contained in the union of hyperplanes. \(\int A = \emptyset\) and
\(y \in \int P \setminus A \neq \emptyset\).

\(y\in \int P\) and \(x_0\in P' \setminus P\). The segment \([y, x_0] \cap
    \partial P = \{z\}\) The segment \([y, x] \nsubseteq A\).

\([y, x_0] \cap A \{x_0\}\) and \(t \in [y, x_0] \cap A\), \(t \neq x_0\). Because
the boundary of \(P\) is the union of faces. Then \(z\) belongs to union of
proper faces. \(z\) cannot be in any face of dimension \(\le n-2\) because \(z\)
would be then in \(A\), then the whole line would be in \(A\). Thus \(z\) belongs
to a facet. Now the supporting hyperplane of this facet would separate \(x_0\)
and \(P\). \(z\) belongs to a facet \(F = P \cap H\), and \(P \subset H^{-1}\) and
\(x_0 \in \in H^{+}\). Contradiction with \(x_0 \in P' = \cap H^{-1}\).
\subsection{Corollary}
\label{sec:org66ebec0}
\(P = \cap H^{-}\), where \(H\) is a supporting facet. Moreover, this is a
minimum representation, we cannot drop any half space from this
representation.
\subsection{Perspective}
\label{sec:org3219268}
\begin{enumerate}
\item We want to study the boundary of \(P\) which is the union of faces.
\item We know that it is the union of finitely many faces.
\item \(F(P)\) is the set of all faces of \(P\). \((F(P), \subset)\) is a partially
ordered set.
\item We can also write the Polyhedra as a disjoint union of the relative
intersection of the faces.
\end{enumerate}
\subsection{Theorem}
\label{sec:org44831df}
If \(P\) is a bounded polyhedral set, then it is a Polytope.
\subsubsection{Proof}
\label{sec:org4c5bbc1}
Induction on the dimension. \(P\) be an \(n\) dimensional bounded polyhedral set
in \(\R^n\). Then all faces of \(P\) are polytopes by induction hypothesis.
(face is the intersection of polyhedra with a hyperplane.) It follows that
\(P \cap H\) is a polytope. \(\conv(\cap \relint F)\)

\(\cup F \subset F \implies \conv(\cap F) \subset P\).

Now we will prove that the interior of \(P\) is contained in the \(\conv(\cup
    P) \subset P\). \(x\in \int P\), then \(x \in \conv(F_1 \cup F_2) \subset \conv
    (\cup F)\). The argument is something about taking the closure.\footnote{Reminds me of something form Topology.}
\subsection{Lemma}
\label{sec:orgeb8f516}
If \(F_1\) and \(F_2\) are faces of \(P\), \(F_1\) is subset of \(F_2\), then we prove
that \(F_1\) is a face of \(F_2\). We then gave a counterexample to the "inverse"
statement, but we claim that this would be true if we work with Polytopes.
\subsection{Theorem}
\label{sec:org226c8a2}
Let \(P\) be a polytope (\footnote{It also holds for Polyhderal set. But we didn't do it.}) and \(F_n\) is a face of \(P\) and \(F_2\) is a
face of \(F_1\), then \(F_2\) is a face of \(P\).
\subsubsection{Proof}
\label{sec:org1989a94}
\(F_1 \le P \implies\) There is a hyperplane \(H_1 = \{x \vert \langle x, u_1
    \rangle = 0 \}\) such that \(F_1 = P \cap H_1\) and \(P\) is a subset of
\(H_1^{+}\). \(F_2 \le F_1 \implies\), there is a hyperplane \(H_2 = \{x \in H_1
    \vert \langle x, u \rangle = 0\}\) such that \(F_2 = F_1 \cap H_2 \subset H_1\)
and \(F_1 \subset H_2^{+}\).

We need a hyperplane that cuts \(P\) and only cuts in \(F_2\).

\(H(\eta) = \{x \in \R^n \vert \langle x, \eta u_1 + u_2 \rangle = 0\}\), \(F_2
    \subset H(\eta)\)

\(\eta_0 = \max\{-\frac{\langle w, u_2\rangle}{\rangle w, u_2\rangle} \vert w
    \in \textup{vert} P \ \textup{vert} F_1 \}\). This is a finite set.

Let \(\eta > \eta_0\) and we prove that \(H(\eta) \cap P = F_2\) and \(P \subset
    H^{+}(\eta)\).

\begin{enumerate}
\item \(w\in \textup{vert} P \setminus \textup{vert} F_1\). \(\langle w, \eta
       u_1 + u_2 \rangle > \eta_0\), \(\langle w, u_1\rangle + \langle w, u_2
       \rangle \ge 0\), implies that \(w\in H^{+}(\eta)\).
\item \(w \in \textup{vert} F_1 \setminus \textup{vert} F_2\) \(\langle w, \eta
       u_1 + u_2\rangle = \eta\langle u, u_1 \rangle + \langle w, u_2 \rangle =
       \langle w, u_2 \rangle \ge 0\).
\item \(w\in \vert F_2\), \(\langle w, \eta u_1 + u_2 \rangle = 0\).
\end{enumerate}
\subsection{Theorem}
\label{sec:org40a7e16}
Any proper face of a polytope \(P\) is a face of a facet.
\subsubsection{Proof}
\label{sec:org052f4f7}
\begin{enumerate}
\item \(x\in F\) where \(F\) is a proper face of \(P\), then \(x\) belongs to a facet.
\(\dim F \le n-2\) and \(y\in \int P\) \(M = \cup \aff(\{y\} \cup G)\). Here
\(G\) is a a face of dimension

\(z\in B(x, \varepsilon) \cap (\R^n \setminus M) \cap (\R^n \setminus P)\)

\([z, y] \cap \partial P = \{t\}\) and \(t_\varepsilon\) is a facet of \(P\).

We create a sequence of points on the facet that converges to \(x\). Each
of the point belongs to a facet, but need to be the same. But since there
are only a finite number of facets, there is a subsequence that is
contained in exactly one facet and converges to \(x\).
\item \(F\) is included in a facet. \(x\in \relint F\), by \((1)\), we have that \(x\)
is an element of \(F_0\), where \(F_0\) is a facet of \(P\). \(F_0\) is a
polytope and there exists \(F'\) a face of \(F_0\), where \(x\in \relint F'\).

\(F \le P\) and \(F_0 \le P\) and \(F' \le F_0\), implies that \(F' \le P\). \(x
       \in \relint F \cap \relint F'\), implies that \(F = F' \le F_0\).
\end{enumerate}
\subsection{Theorem}
\label{sec:org98d5ce7}
Let \(\dim F' = 1\), where \(F_0\) is a face of an \(n\) polytope \(P\). If \(0 \le i
   < k\) and \(F^i \le F^k\), then there exist faces \(F^{i+1}, \cdots, F^{k-1}\)
such that \(F^i\) is contained in \(F^{i+1} \subset F^k\).
\subsubsection{Proof}
\label{sec:orgab61040}
Induction on \(k\), when \(k=i+1\), there is nothing to prove. Assume \(k > i+1\),
then \(F^i \subset \cdots \subset F^k \subset F^k\).
\section{Lecture 12 \textit{<2018-11-27 Tue>}}
\label{sec:org1be1f0b}
\subsection{Partially Ordered sets}
\label{sec:org055863a}
Let \(P \neq \emptyset\) and \(\rho \subset P^2\) is called a binary relation.

\begin{enumerate}
\item \(\rho\) is reflexive, if the diagonal \(\Delta = \{(x, x) \vert x \in P \}
      \subset \rho\)
\item \(\rho\) is symmetric if \((x, y) \in \rho\), then \((y, x) \in \rho\).
\item \(\rho\) is anti symmetric if \((x, y) \in \rho\) and \((y, x) \in \rho\), then \(x = y\)
\item \(\rho\) is transitive if \((x, y) \in \rho\) and \((y, z) \in \rho\), then \((x,
      z)\in \rho\).
\end{enumerate}

We then define the notion of a relational equivalence if is reflexive,
symmetric and transitive. Then we define the notion of equivalence classes
and partition of a set. An example is in Groups. There is also an equivalence
between the partitions of \(X\) and relations of equal in \(X\).
\subsection{Partial order}
\label{sec:orgb2901f5}
\(\rho\) is \textbf{partial order} if \(\rho\) is Reflexive, Anti symmetric and
transitive.

Examples: \((\N, \le)\), \((\N, \vert)\). \((\Z, \vert)\) is not a partial order.

\((\R, \le), (\R \times \R, \le)\) where \(\le\) is defined by \((x, y) \le (z, a)
   \iff (x \le z) \textup{and} y \le a\)

\(G\), \(\operatorname{sub}_G =\) subgroups of \(G\)

\(X \neq 0\), \((P(X), \subset)\). Let \((P, \le)\) be a partial order, then if
\begin{enumerate}
\item If for every \(p, q \in P\) holds and \(p \le q\) or \(q \le p\), then \(\le\) is
called a linear order.
\item An element \(m\) is the minimum of \(P\) if for every \(p \in P\), the following
holds: \(m \le p\). The minimum element is unique. This is because of the
anti-symmetric.
\item An element is called the minimal element, if there is no element smaller
than the minimal element. A minimum is clearly a minimal element. For
example if we take \(\N\) without \(1\) and the relation being divisibility,
there is no minimum any more, but every primes are minimal elements.
\item One can define maximal elements similarly. Also similarly for maximum.
\end{enumerate}

A \textbf{cover relation} of order \((X, \le)\) is the relation \(\precsim\) define by \(p
   \precsim q \iff\) for all \(r \in P\), \(P \le r \le q \implies p = r\) or \(q =
   r\).

A \textbf{Hasse diagram} is the following thing: take the minimal elements at the
bottom., then we draw an arrow from the bottom element to the top element if
\(p \precsim q\) and similarly one can continue this process.

An example with \((\Z, \vert)\) was drawn. Another example, subgroups of \(\Z\).
It turns out that the Hasse diagram is identical to \((\Z, \vert)\).

A map \(f\colon P \rightarrow Q\) between orders is a \textbf{monotone map} (\textbf{order
preserving}) if \(x \le y \implies f(x) \le f(y)\). We can also similarly
define a \textbf{order reversing}.

\textbf{Remark} Partial Ordered sets with order preserving maps form a Category.

\subsection{Boundary of Polytope}
\label{sec:orgaa74d46}
Boundary of Polytope: There are finitely many faces of \(P\).

\(B(P) =\) the set set of all faces of \(P\).

\(B_0(P) =\) the set of all proper faces of \(P\). \(B_0(P)\) does not contain the
empty set and \(P\).

\((B(P), \subset)\) is a finite order with minimum \(\emptyset\) and maximum \(P\).

Another Property is that \(P = \conv(\operatorname{vertices} P)\). We also know
that the Polyhedral set is equal to the finite intersection of closed half
spaces.

If \(P\) is a bounded Polyhedral set, then \(P = \cap H^{-1}\) where \(H\) is a
supporting hyperplane of a face of \(P\).

We also know that \(F_1 \subset P\) and \(F_0 \le F_1\), then \(F_0 \le P\). In
terms of Hasse diagram, if \(F_1 \subset P \implies B(F_1) \subset B(P)\)

Also \(F_0 \subset P \implies\) there exists a face of \(P\), \(F_0 \le F\). The
maximal elements are facets and the minimal elements are vertices.

\(F^j \le F^k \le P\) and \(j \le k\), then there exists. \(F^i \le F^{j+1} \le
   F^{j+2} \le \cdots \le F^{k-1} le F^k\).

Recall the definition of a ridge. It's an \(n-2\) dimensional face of \(P\).
\subsection{Theorem about ridges}
\label{sec:orgba52580}
Every \(\dim P - 2\) face of \(P\) is the intersection of exactly two
facets.\footnote{In algebraic topology, we know that the boundary of a Polytope is a CW
complex. What we're proving right now is that the boundary is a Pseudo Manifold.}
\subsubsection{Proof}
\label{sec:org33cd19c}
\(F_1, \cdots, F_s\) are facets of \(P\)

\(F_1 = P \cap H_i\) and \(P \subset H_i^{-}\), \(\dim F = \dim P - 2\), \(P =
    \cap_{i=1}^{s} H_i^{-}\), \(F_j = P \cap H_j = \cap_{i=1}^{s} H_i^{-} \cap H_j
    = \cap_{i \neq j} H_i^{-} = \cap_{i \neq j}(H_i^{-} \cap H_j)\). Notice that
\(H^{-1} \cap H_j\) are closed half spaces in \(H_j\). So \(F\) a facet of \(F_j\)
is of the form \(F = F_j \cap H_r\) for \(1 \le r \le s\).

\(F = F_j \cap H_r = (P \cap H_j) \cap (H_r \cap P) = F_j \cap F_r\).

The exactly part is missing. Let us assume the contrary, that a ridge is in
the intersection of three facets. Let us not take a point in the interior
and define a hyperplane that would cut the three planes and form three
lines. Now one of the lines have to be in the convex hull and (the trick is
to use the fact that the dimension of a facet is \(n-1\), thus we may think
about the orthogonal complement.) The contradiction is that the facet will
have a hyperplane that will separate the two parts of the Polytope. But in
our case, this cannot happen and hence a contradiction. \footnote{For manifolds this proves that the top homology is exactly \(\Z\).}
\subsection{Theorem}
\label{sec:orgc5d2926}
Let \(P\) be a Polytope and \(1 \le j \le k < \dim P\), then every \(j\) face \(F^j\)
is an intersection of \(k\) faces of \(P\). In particular, each proper face is an
intersection of facets.
\subsubsection{Proof}
\label{sec:org2ae2f2e}
\(F^j \le P\) and \(F^j \le F^{j+1} \le \cdots \le F^{k+1} \le \cdots \le P\).

There are two cases

\begin{enumerate}
\item If \(k = j+1\) and \(k +1 = j+2\), then \(F^j \le \le F^{j+2}\). Then \(F^j\) is
a ride of \(F^{j+2}\), so by the above theorem, \(F^j\) is the intersection
of two facets of \(F^{j+2}\) that are dimension \(j+1 = k\) faces of
\(F^{j+2}\) and consequently the faces \(P\).
\item If \(k > j +1\), then \(F^i \le F^{j+2} \le \cdots F^{k+1}\). Therefore \(F^j\)
is the intersection of two \(j+1\) faces if \(k > j+2\) and we continue doing
this until \(k = j+2\).
\end{enumerate}
\subsection{Definitions}
\label{sec:orgcf13695}
\(B_0(P)\) all proper faces: reduced boundary complex

\(B(P)\) boundary complex.

\(\psi\colon B(P) \rightarrow B(Q)\), then \(P \equiv Q\) are combinatorially
equivalent.
\section{Lecture 13 \textit{<2018-11-27 Tue>}}
\label{sec:org65e30da}
\subsection{Recap}
\label{sec:org4045857}
\(P\) is a polytope of dimension \(n\) in \(\R^n\).

\(B_0(P)\) is the family of all proper faces of \(P\) which is equal to the
reduced boundary complex of \(P\).

\(B(P)= B_0(P) \cap \{\emptyset\}\) is the boundary complex of \(P\).

\(L(P) = B(P)\cap \{P\}\) is the face lattice of \(P\)

All these sets form partially ordered sets with respect to inclusion. \(B_0(P)
   \subset B(P) \subset L(P)\).

All of this happen inside the category of all partially ordered set. The
objects of the category are all pairs \((P, \le)\) where \(\le\) is an order on
\(P\).

The morphisms \(M(P, Q)\) are functions \(\{f \colon P \rightarrow Q \colon p_1
   \le p_2 \implies f(p_1) \le f(p_2)\}\).

A functor from Posets to Posets. \((P, \le)\) to \((P^{*}, \rho)\). Where \(P^{*}
   = P\) and \(p_1, p_2 \in P{*}\), \(p_2 \rho p_2 \iff p_2 \le p_1\). We basically
reverse the order on the first partial order. We also need to tell how the
functor maps morphisms to morphisms. Given a morphism, \(f\colon P
   \rightarrrow Q\), what is the map \(F(f) \colon P^{*} \rightarrow Q^{*}\). We
define \(F(f) = f\).
\subsection{Definition}
\label{sec:org4a8029f}
Let \(P\) be an \(n\) Polytope and \(f_i(P)\) is the number of \(i\) dimensional
faces of \(P\), then the \(f\) vector of \(P\) is \(f(P) = (f_0(P), f_1(P), \cdots,
   f_{n-1}(P))\). We set \(f_{-1}(P) = 1\) and \(f_n(P) = 1\).
\subsection{Remark}
\label{sec:orge7c098d}
What vectors in \(\N^n\) are \(f\) vectors of \(n\) polytopes? We should come up
with characteristics for \(f\) vector.

A general characterization is not known. But we can look at subfamilies.

Look at the subfamily of Simplicial complexes. \href{https://en.wikipedia.org/wiki/Simplicial\_complex}{Wikipedia}
\subsection{A conjecture by Peter McMallen}
\label{sec:org3dc4d79}
\((f_0, f_1, \cdots, f_n) \in \N^n\) is a \(f\) vector of simplicial \(n\)
polytopes if and only if 

\begin{enumerate}
\item Dehn-Somoville equations.
\item A bunch of inequalities.
\end{enumerate}
\subsection{Dual Polytopes and Quotient Polytopes}
\label{sec:org82b8083}
\(P\) is a polytope. \(L_(P)\) is a lattice. What is a lattice? Every fininte
subsets have an infimum and supremum.

The functors give the following commutative diagram. Arrows go down and
right.
\begin{verbatim}
P---------------------------- [ ]
|                              |
|                              |
|                              |
|                              |
|                              |
L(P)-------------------------L(P)^*
\end{verbatim}

\(P_1 \equiv P_2\iff L(P_1) \equiv L(P_2)\)

I think the idea is that \(L(P_1)\) is all about combinatorics. So if they are
combinatorially isomorphic, \(P_1\) and \(P_2\).

If \(P\) is a polytope, \textbf{the dual polytope} is a polytope \(P^{*}\) whose face
lattice is dual of \(L(P)\). The dual Polytope is a class, not just a single
Polytope, unlike the Polar of the Polytope which is a polytope. \footnote{Apparently was asked in the Oral exam.}
\subsection{Polarity in \(\R^n\)}
\label{sec:org96743c3}
Polarity in \(\R^n\) with respect to the unit sphere.

\(0 \notin W\) the affine space \(\mapsto \pi(W) = \cap_{u \in W} H_{u} =
   \cap_{w \in W}\{w \vert \langle x u \rangle = 1\}\) The claim is that \(\pi(W)
   = u - 1 - \dim W\).

\(P\) is an affine space of dimension \(0 \mapsto \pi(p) = \cap H_u = H_p = \{x
   \vert \langle x, up \rangle = 1\}\)

\(K\) a convex body with \(0 \in \int K\), \(K^{*} = \cap_{u \in K} H_u^{-} =
   \cap_{u\in \partial K} H^{-}_u\), the polar body.
\subsection{Theorem}
\label{sec:orga1fc7fe}
Let \(P\) be an \(n\) dimensional Polytope in \(\R^n\) and \(0\) is an element of the
interior of \(P\). Let \(P^{*}\) be the polar under the Polarity \(\pi\).

\begin{enumerate}
\item If \(F\) is a proper face of \(P\), then \(F{*} = P^{*} \cap \pi(\aff F)\) (by
definition) and the \(\dim F^{*} = n - \dim F - 1 = \dim \pi(\aff F)\)
\item Assignment \(F\) to \(F^{*}\) for all \(F \in B_0(P)\) induces a \(\phi\), an
order reversing map \(\varphi \colon B_0(P) \rightarrrow B_0(P^{*})\).
\item \(F^{**} = F\).
\item \(P^{*} = \cap_{u \in \vert P} H^{-}_u\). In particular, \(P^{*}\) is an \(n\)
dimensional polytope with \(0 \in \int P^{*}\).
\end{enumerate}

Then this will be a dual Polytope.
\subsubsection{Proof}
\label{sec:org383d71a}
\begin{itemize}
\item (Proof of 4) \(P^{*} = \cap_{u \in P} H^{-1}_u = \cap_{u \in P} \{x \vert \langle x, u
      \rangle \le 1\}\), \(P^{*} = \cap_{u \in \partial P} H^{-1}_u\) and \(\vert P
      \subset \partial P\) and

\(K\) is a convex body with \(0 \in \int K\) and \(K^{*} = \cap_{u \in K}
      H^{-}_u = \cap_{x \in partial K} H^{-}_u\).

\(P^{*} \subset \cap_{u \in \vert P} H^{-}_u \implies \forall u \in
      \partial P, x \in H^{-1}_u\).

Now we know that \(P\) is a convex hull of vertices and 

\(x \in \cap_{u \in \vert P} H'_u \implies \langle x, x_1 \rangle \le 1,
      \cdots, \langle x, x_s \rangle \le 1\)

\(u\in \partial P \implies \exists \lambda_1, \cdots, \lambda_s \ge 0\), \(u
      = \sum \lambda_i x_i\), such that \(\sum \lambda_i = 1\).

\(\langle x_i, u \rangle = \langle x, \lambda_1 x_1 + \cdots, \lambda_s
      x_s \rangle = \lambda_1\langle x, x_1\rangle + \cdots + \lambda_s \langle
      x, x_s\rangle \le \lambda_1 + \cdots + \lambda_s = 1\).

Thus \(x \in H^{-}_u\).

\(B(0, r) = \{x \vert \Vert x \Vert \le 1\}\)

\(B(0, r)^{*} = B(0, \frac1r)\).

\(0 \in \int P \implies \exists r > 0, B(0, r) \subset \int P \subset P\).

\(K_1 \subset K_2\), then \(K_2^{*} \subset K_1^{*}\).

\(P^{*} \subset B(0, r)^{*} = B(0, \frac1r) \implies P^{*}\) is bounded.

\(P\) is a polytope \(\implies L >0, P \subset B(0, L) \implies B(0, L)^{*}
      \subset P^{*} \implies B(0, 1/L) \subset P^{*}\)

\item (Proof of 1) \(F^{*} = P^{*} \cap \pi(\aff F) = (\cap_{u \in \vert P}
      H_u^{-} \cap \pi(\aff F) = \cap_{v \in \vert P} H^{-1}_u \cap \cap_{u \in
      \{x_1, \cdots, x_r\}} H_u\)

\(\pi(W) =\cap_{w\in W} H_u = \{x \vert \forall u \in W \langle u, x
      \rangle = 1\}\).

\(\dim W = p \implies y_p, \cdots, y_n \in W\) affinely independent such that
\(y\in W \implies \exists \lambda_0, \cdots, \lambda_p)\), \(y\in \sum
      \lambda_i y_i, \sum \lambda_i = 1\).

\(\langle y, x \rangle = \sum \lambda_i \langle y_i, x \rangle\).

Going back, 

\(\cap_{u \in \{x_1, \cdots, x_r\}} H_u^{-1} \cap \cap H_u = \cap H_u \cap
      \cap H^{-}_u = \cap (L \cap H^{-}_u)\)

(A simple argument), \(F^{*} = P^{*} \cap \pi(\aff F) = P^{*} \cap \cap_{u
      \in \ver F} H_u = \cap_{u \in \int F \subset \partial P} (P^{*} \cap H_u)\)
This is a supporting Hyperplane for the dual.

Lemma: \(u \in \partial P \implies H_u\) is a supporting hyperplane of
\(P^{*}\). \(F^{*} = P^{*} \cap H_u = \cap(H_u \cap P^{*})\) The intersection
of faces is a face.

What about the dimension? \(\dim F^{*} = n - \dim F - 1\). Thus \(F^{*} =
      P^{*} \cap \pi(\aff(F)) \subset \pi(\aff F)\).

The \(\dim F^{*} \le \dim \pi(\aff F) = n - \dim F - 1\).

Now \(F = \cap_{i = 1} F_i\) where \(F_i\) are facets. \(F_i = P \cap H_{u_i}\)
where \(u_i\) are some vectors.

\(t = \textup{codim} F \implies u_1, \cdots, u_t\) are affinely independent.
Now we want to prove that \(u_i \in F_i^{*} \subset F^{*}\). If this is
true, then it would mean that \(\dim F^{*} \ge t - 1 = n - \dim F - 1\). In
order to prove this, we can say that \(F_i = P \cap H_i\) is a facet. Now
\(F_{i}^{*} = P^{*} \cap \cap \pi(H_{u_i}) = \cap H_{u \in \ver P}^{-} \cap
      \cap_{x\in H_{u_i}} H^{-1}_u\). \(u_i\) belongs to every single one of them,
therefore belongs to the intersection. There something else that I missed.

\item \(F^{**} = F\). We know that \(F = P \cap H = P^{**} \cap \pi(\pi(\aff F)) =
      P^{**} \cap \pi(\aff F^{*}) = F^{**}\). If you prove this, you prove that
your map \(\phi^2 = \id\), then \(\phi\) is bijection.
\end{itemize}
\section{Lecture 14 \textit{<2018-12-05 Wed>}}
\label{sec:orgce705b5}
\subsection{Dual and quotient polytopes}
\label{sec:org3a1d259}
\textbf{A dual polytope} \(P^{*}\) of the polytope \(P\) is any polytope, whose face
 lattice is anti-isomorphic to the face lattice of \(P\).

\(P\) is a polytope and \(F(P) = B(P) \cap \{P\}\) is a partially ordered set.
We don't know if it is a lattice.

Anti-isomorphic. Meaning we have a bijection between the face lattices that
reverses the inclusion.

Here \(P^{*}\) is a class of Polytopes (Polar is one element in the class.)
\subsubsection{Example}
\label{sec:org34ff9e5}
A prism in \(\R^3\). \(5\) facets. Hence the dual should have 5 vertices.

The edges should be edges. We join the \(5\) points in the dual similarly.
\subsubsection{Polar body}
\label{sec:org73fdc7f}
If \(K\) is a convex body in \(\R^d\) with \(0 \in \int K\) \footnote{We know that the body is full dimensional because we have an interior point.}, then \(K^{*} =
    \{ y \vert \forall x \in K, \langle x, y \rangle \le 1\}\).
\subsubsection{Lemma 1 (Polar body)}
\label{sec:org9fc7d4d}
\(K^{*}\) is a closed set.

The proof kinda depends on the fact that \(K^{*]\) can be written as the
intersection of \(H_x^{-}\). Now these sets are closed. \(K^{*}\) is closed
since it is intersection of closed sets.
\subsubsection{Lemma 2 (Polar body)}
\label{sec:orgfd4a8bd}
If \(K_1 \subset K_2\), then \(K_2^{*} \subset K_1^{*}\).

The proof is kinda easy.
\subsubsection{Lemma 3 (Polar body)}
\label{sec:org2634fb8}
\(K^{*}\) is bounded and \(0 \in \int K^{*}\) (Also tells us that \(K^{*}\) is of
maximum dimension)

We need to prove something about the Polar body of the closed ball, i.e.,
\(B(0, r)^{*} = B(0, 1/r)\).

\(y \in B(0, r)^{*} \implies \forall x \in B(0, r), \langle x, y \rangle \le
    1\). Hence \(r y \frac{\Vert y \Vert} \in B(0, r)\), now \(\langle r y/{\Vert y
    \Vert}, y \rangle \le 1 \iff \frac{1}{\Vert y \Vert} \iff \Vert y \Vert \le
    1/r \implies y \in B(0, 1/r)\). Showing the other part is also easy.

Now K convex body \(\implies\) K compact \(\implies\) K bounded, implies that \(K
    \subset B(0, r)\). Now \(0 \in B(0, 1/r) \subset K^{*}\) and thus \(0 \in \int
    K^{*}\).
\subsubsection{Lemma 6 (Polar body)}
\label{sec:orgc02debc}
\(K^{**} = K\)

\(x \in K\) and \(y \in K^{*}\), then \(\langle x,y \rangle \le 1\) for all \(y \in
    K^{*}\).

\(K^{**} = \{z \vert y \in k^{*}, \langle z , y \rangle \le 1\}\)

\(K \subset K^{*}\)

\(x_0 \in K^{**} - K\), and \(H_y = \{x \vert \langle x, y \rangle = 1\}\), \(K
    \subset H_y^{-}\) and \(x_0 \in \int H_y^{+}\), \(\forall x\in K, \langle x, y
    \rangle \le 1 \implies y \in K^{*}\).

\(x_0 \in \int H_y^{+} \implies \langle x_0, y \rangle > 1\) and \(x_0 \notin
    K^{**}\) which is a contradiction.
\subsubsection{Lemma 7 (Polar body)}
\label{sec:org3215700}
Let \(F\) be a face of \(K\) and set

\(\hat{F} = \{ y \in K^{*} \vert \forall x\in F, \langle x, y \rangle = 1\}\).

Then \(\hat{F}\) is a face of \(K^{*}\).
\begin{enumerate}
\item Proof
\label{sec:org7febb01}
What is the hat of the empty set? It is \(K^{*}\).

\(\hat{K}\), this has to be empty because \(K\) contains \(0\) and there is no
element \(\langle y, 0 \rangle = 1\).

\(F\) be a proper face and \(x_0 \in \relint F\), Define (not really the dual)
\(F^{*} = \{y \in K^{*} \vert \langle y, x_0 \rangle = 1\} = K^{*} \cap \{y
     \vert \langle y, x \rangle = 1\}\). This is a face. Here \(F\) is only true
for one \(x_0\), but \(\hat{F}\) should be true for every \(x_0\). Thus \(\hat{F}
     \subset F^{*}\).

Now we need to prove the other side, i.e., \(F^{*} \subset \hat{F} \subset
     K^{*}\). (One should prove that) If \(x \in F^{*} \implies x \in \hat{F}\). Or
\(x \in K^{*} \setminus \hat{F} \implies x \in K^{*} \setminus F^{*}\).

\(x \in K^{*}\setminus F\implies \exists x_1 \in F, \langle x, x_1 \rangle <
     1\), now \(x_0 \in \relint F\), this means that we can draw a segment between
\(x_0\) and \(x_1\) and \(x_2\) be an element such that \(x_0 = (1-\lambda)x_1 +
     \lambda x_2\),

Now we should show that \(\langle x, x_0\rangle \neq 1\). Now \(\langle x, x_0
     \rangle = \langle x, (1-\lambda)x_1 + \lambda x_2\rangle =
     (1-\lambda)\langle x, x_1 \rangle + \lambda \langle x, x_2 \rangle\). We
know that \(\langle x, x_1 < 1\) and \(\langle x, x_1 \rangle \le 1\). All
together, we see that \(\langle x, x_0 \rangle < 1\), which is what we wanted
to prove.

Thus \(F^{*} = \hat{F}\) and \(F^{*}\) is a face implying \(\hat{F}\) is a Face
of \(K^{*}\).
\end{enumerate}
\subsubsection{Lemma 8 (Polar body)}
\label{sec:org2a40a7f}
The correspondence \(F \mapsto \hat{F}\) between \(F(K)\) and \(F(K^{*})\) (Here
\(K^{*}\) is the polar body) is bijective and inclusion reversing.
\begin{enumerate}
\item Proof
\label{sec:orgc58f8e9}
\(\hat{F} = \{y \in K^{*} \vert \forall x \in F, \langle x, y \rangle =
     1\}\).

\(F_1 \subset F_2\) faces of \(K \implies \hat{F}_2 \subset \hat{F}_1\). This
is clear?

To complete, it suffices to prove that \(\hat{\hat{F}} = F\)

It suffices to prove that \(\hat{\hat{F}} = \{ x\in K^{**} \vert \forall y
     \in \hat{F}, \langle x, y \rangle = 1\}\) and \(\hat{\hat{F}} = \{x \in
     K\vert \forall y \in \hat{F}, \langle x, y \rangle = 1\}\supset F\)

We need to prove that \(\hat{\hat{F}} \subset F \subset K\).

Now \(x_0 \in K\setminus F, \implies x_0 \in K\setminus {\hat{\hat{F}}\)

\(x_0 \in K \setminus F \implies x_0 \in K\setminus {\hat{\hat{F}}\), \(x_0
     \in \int H_y^{-} \implies \langle x_0, y \rangle < 1\).

\(F = K \cap H_y = \{x \in K \vert \langle x, y \rangle = 1\} \implies y \in
     \hat{F} \implies x_0 \notin {\hat{\hat F}}\)
\end{enumerate}
\subsection{Theorem}
\label{sec:orgf828934}
Let \(P\) be a \$d\$-polytope in \(\R^d\) with \(0 \in \int P\), then the polar
\(P^{*}\) of \(P\) is a \$d\$-polytope dual to \(P\).
\subsubsection{Proof}
\label{sec:orga2bbb02}
We've already shown that the face lattice of \(P^{*}\) is anti-isomorphic to
the face-lattice of \(P\). But we don't know if \(P^{*}\) is a Polytope.

We also know that \(P\) is convex body. This means that \(P^{*}\) is a convex
body, in particular, this means that \(P^{*}\) is bounded. Now it suffices to
prove that it is the intersection of half spaces.

Let \(Q \subset \cap \{y \vert \langle x, y \rangle \le 1\}\) a polyhedron.
\(P^{*} \subset Q\) since \(\vert P \subset P\).

It remains to prove that \(Q \subset Q^{*}\). \(\vert P = \{v_1, \cdots, v_k\}\)
and \(x\in Q\), \(\langle x, v_1 \rangle \le 1, \cdots, \langle x, v_k \rangle
    \le 1\).

\(P = \conv \{v_1, \cdots, v_k\}\implies \forall y \in P, \exists \lambda_1,
    \cdots \lambda_k \ge 0, y = \sum \lambda_i v_i, \lambda_i = 1\), for
arbitrary \(y \in P\), we should prove that \(\langle x, y \rangle \le 1\). Now
\(\langle x, y \rangle = \langle x, \sum \lambda_i v_i \rangle = \sum
    \lambda_i\langle x, v_i\rangle \le \sum \lambda_i \le 1\).

Thus we can conclude that \(x\in P^{*}\), thus \(P^{*}\) is a polyhedron. We
already proved that \(P^{*}\) is bounded, thus it is a Polytope.
\subsection{Remark (What about lattice?)}
\label{sec:org45858a6}
The family of faces with \(\subset\) a lattice. \((F(P), \subset)\) is a lattice.

In order to be a lattice, every two elements must have an infimum and
supremum.
\subsubsection{Proof}
\label{sec:orgcd66ca7}
\(F_1\) and \(F_2\) be face of \(P \implies F_1 \wedge F_2 = F_1 \cap F_2\) and
\(F_1 \vee F_2 = (\hat{F}_1 \wedge {\hat F}_2)\). This is how we define the
infimum and supremum.
\subsection{Lemma}
\label{sec:orgc30f897}
Let \(P\) be a \(d\) polytope and \(0 \in \int P\). Let \(L\) be an \(n\) dimensional
linear subspace and \(\sqcap \colon \R^d \rightarrow L\) corresponding
orthogonal projection. Then \(L \cap P^{*}\) is the polar set in \(L\) of
\(\sqcap(P)\).

There was a diagram drawn.

For \(L \cap P^{*}\), polar of \(\sqcap(P)\), we can decompose \(\R^d = L \oplus
   L^\perp\) So every element of \(\R^d\) can be thought of as a sum of an element
of \(L\) and an element of \(L\perp\).

\(\vert P = \{v_1, \cdots, v_k}\), \(v_i = w_i + w_i^'\), where \(w_i \in L\) and
\(w_i' \in L^\perp\)

\(L \cap P^{*} = \{y \in L \cap \vert \forall x \in P, \langle x, y \rangle \le 1\}\)

This is equal to \(\{y \in L \vert \langle v_i, y \rangle \le 1\}\).

Now \(\langle w_i, y \rangle + \langle w_i', y \rangle = \langle w_i + w_i', y
   \rangle \le 1\).

This is same as \(\{y \in L \vert \langle w_i, y \rangle \le 1\} = \cap
   H_{w_i}^{-}(\textup{in} L)\) The last sum is \((\conv\{w_1, \cdots, w_k\})^{*}\)
Each \(w_i = \sqcap(v_i)\).

Then it is same as \(\conv \{\sqcap(v_1), \cdots, \sqcap(v_k)\} =
   \sqcap\conv{v_1, \cdots, v_k\} = \sqcap(P)\).
\subsection{Theorem}
\label{sec:org1528ddf}
Let \(F_1\) be a \(j\) face and \(F_2\) a \(k\) face of \(P\) and \(F_1 \subset F_2\),
then the sub lattice of \(F(P)\) consisting of all faces \(F\) such that \(F_1
   \subset F \subset F_2\) is isomorphic to a face lattice of some \(k-j-1\)
polytope we denote by \(F_2/F_1\).
\subsubsection{Proof}
\label{sec:org05743fd}
We have \(F_2, F_1\) and we have a sub lattice. Now consider \([F, F_2]\), take
a polar of \(F_2\), the face lattice changes the position. Now it is a face
lattice of \([\hat{F}_1, \hat{F}_2]\).
\section{Lecture 15}
\label{sec:orga37c780}
\subsection{Review}
\label{sec:orgaa6722f}
The dual of a polytope was discussed.
\subsection{Quotient polytope}
\label{sec:org8280675}
\subsubsection{Proposition}
\label{sec:org0a8e0ac}
Let \(P\) be an \(n\) polytope \(F_1\) a \(j\) face of \(P\) and \(F_2\) be a \(k\) face
of \(P\) such that \(F_2 \supset F_1\). Thus sublattice of all faces \(F\) such
that \(F_1 \subset F \subset F_2\) is isomorphic to a face lattice of a
\(k-j-1\) polytope that we denote by \(F_2/F_1\).
\subsubsection{Some random stuff}
\label{sec:org993e5e1}
\(F_2\) is a face of \(P\), \(F_2\) is a a face of \(P\), \(F_1 \subset F_2\).
\(F_2^{*}\) is a polar of the \(F_2\), \(\hat{F}_1\) dual face of \(F_1\) inside
\(F_2\).

The face lattice of \(\hat{F}_2\) is anti-isomorphic to \([F_1, F_2]\).

\((\hat{F}_1)^{*} \approx [F_1, F_2]\).

We can say that \(F_2/F_1 \approx (\hat{F}_1)^{*}\).
\subsubsection{Corollary}
\label{sec:org6a6b1ec}
If \(\hat{F}_1\), \(\hat{F}_2\) are face of \(P^{*}\). \(F_1 \subset F_2\) and
\(\hat{F}_2 \subset \hat{F}_1\).

\(\hat{F}_1/\hat{F}_2 \approx (F_2/F_1)^{*}\)
\subsubsection{Corollary}
\label{sec:org8078c00}
If \(F_1 \subset F_2 \subset F_3\) are faces of the polytope \(P\). Then
\((F_3/F_1)/(F_2/F_1) \approx F_3/F_2\).
\subsection{Definition (Face figure)}
\label{sec:org44e769f}
Let \(P\) be a polytope and \(F\) a vertex of \(P\) with \(H\) a hyperplane,
separating \(F\) from \(\vert P \setminus \{F\}\), then a vertex-figure of \(P\) at
\(F\) is the polytope \(P \cap H\).

Why is it a polytope. Because we're intersecting a hyperplane with a
polytope. A picture was drawn.

\(\int H^{*} \supset F\) and \(\int H^{-} \supset \ver{P} - \{F}\)
\subsection{{\bfseries\sffamily TODO} Lemma}
\label{sec:orgefba5e4}
A vertex figure of \(P\) at the vertex \(F\) is combinatorially combinatorially
isomorphic to \(P/F\).
\subsubsection{Proof}
\label{sec:org5792879}
Let \(x\) be the vertex \(F\) and \(Q = \conv(\vert P \setminus \{x\})\). Let \(y\)
be the closest point, \(p_Q(x)\).

Then \(H\) can be hyperplane perpendicular to \(y-x\) passing through
\(\frac{1}{2}(x+y)\).\footnote{It's sort of like we choose a hyperplane such that, it passes through
the two vertices that are adjacent to the polytope. It's as if, we create a
polytope where we remove the vertex. Okay, this might be wrong.}

Then \(H\) can be the hyperplane perpendicular to \(y-x\) passing through
\(\frac{1}{2}(x+y)\).

Since \(H\) does not contain any vertex of \(P\), \(H\) intersect every face of
\(P\) containing \(x\) in the relative union.

Let \(H'\) be a hyperplane defining a face of \(P\) that contains \(x\), then
\(H'\cap (H \cap P)\) is the face of \(H \cap P\).

Thus \(H \cap P\) is combinatorially isomorphic to \(P/x\).
\subsection{Construction of quotient polytope using vertex figure}
\label{sec:org08d3686}
\(\emptyset = F^{-1} \subset \cdots \subset F^{j-2} \subset F^{j-1} \subset
   F_1 \subset F_2\)

\(F_2/F^{i}\) is combinatorially equivalent to the vertex figure of
\(F_2/F^{i-1}\) at \(F^{i}/F^{i-1}\). This is an inductive definition.

\(F_2/F^{-1} = F_2\).

\(F_2/F^{0}\) is a vertex figure of \(F_2\).
\subsection{Special types of Polytopes}
\label{sec:org87704ba}
\subsubsection{Simplex}
\label{sec:org06422dc}

Simplex is the convex hull of affinely independent points.

A \(d\) dimensional simplex \(T^d\), is the convex hull of \(d+1\) affinely
independent points.

0 dim: point
1 dim: edge
2 dim: triangle
4 dim  tetrahedron

\(\vert T^d = \{x_0, \cdots, x_d\}\), \(x_0, \cdots, x_d\) are affinely independent.

\(\sum_0^d \lambda_i x_i' \in  \conv\{x_0, \cdots ,x_d\}\)

Faces of \(T^d\). Facets of \(T^d\), \(x_0, \cdots, x_d\) affinely independent
and \(H_d = \aff\{x_0, \cdots, x_{d-1}\}\) hyperplane supporting hyperplane.

\(H_d \cap T^d = \conv\{x_0, \cdots, x_{d-1}\} \approx T^{d-1}\).

Now we can apply induction. This way every subset of vertices is a face of y
polytope.

\(F(T^d) \approx P(\ver T^d) = \textup{Boolean lattice} \approx B_{d+1}\).

\(f_k(T^d) = \binom{d+1}{k+1}\), \(0 \le k \le d\).
\subsubsection{Simplicial}
\label{sec:org50c642d}
A polytope \(P\) with all proper faces being simplices is called simplicial. A
polytope \(P\) where dual \(P^{*}\) is simplicial is called simple.

An example of a simplicial polytope is a bi-tetrahedron. An example of a
simple is a cube.
\subsection{Lemma}
\label{sec:orgefe4c26}
An \(n\) polytope is simple if and only if each vertex is contained in exactly,
\(n\) edges.
\subsubsection{Proof}
\label{sec:orgf2522d0}
\(P\) is a simple polytope \(\implies P^{*}\) is simplicial.

\(v\) is a vertex and \(e_1, \cdots, e_m\) edges, containing \(v\), \(\implies
    \hat{v}\) is a face of \(P^{*}\) and \(\hat{e}_1, \cdots, \hat{e}_m\) are ridges
of \(P^{*}\) containing \(\hat{v}\), or \(\hat{e}_1, \cdots, \hat{e}_m\) are
facets of \(\hat{v}\). Simplex of dimension \(n-1\).

\(P\) is a polytope with such vertex contained in \(n\) edges, implies that
every face of \(P^{*}\) contains exactly \(n\) ridges. This implies every facet
\((n-1)\) dimensional polytope of \(P^{*}\), \(n\) facets.
\subsection{Definition}
\label{sec:org468bdbb}
Let \(Q\) be an \(n-1\) dimensional polytope in \(\R^n\) and \(v\notin \aff{Q}\). The
polytope \(P = \conv(Q \cap \{v\})\) is a pyramid with base \(Q\) and apex \(v\).

\(d-1\) \textbf{fold} of a pyramid is the \(d\) face of \(d\) pyramid which is a simplex.
\subsection{Lemma}
\label{sec:org98ca648}
\(f_k(P) = f_k(Q) + f_{k-1](Q)\), \(0 \le k \le n\).

\(f_{-1}(Q) = f_{n-1}(Q)=1\), \(F_i(Q) = 0\) for \(i < -1\) and \(i > n-1\).
\subsubsection{Proof}
\label{sec:org40c9881}
Let \(H\) be a supporting hyperplane of \(P\). Then \(v \notin H\), \(H \cap P = H
    \cap Q\) is a face of \(Q\).

\(v \in H\) and \(H \cap (\ver P - v)\) gives a face in \(Q\), \(H \cap P = \con
    (\{v \} \cup\) cone of a face of \(Q \cap H\).
\section{Lecture 16 \textit{<2018-12-12 Wed>}}
\label{sec:orgf119c56}
\subsection{Special types of polytopes}
\label{sec:orge569bc8}
\subsubsection{Simplex}
\label{sec:orgef9f444}
A convex hull of affinely independent points.

\(f_k(T^d) = \binom{d+1}{n+1}\)

\(F(T^d) = P([d+1])\)

\([d+1] = \{1, \cdots, d+1\}\).
\subsubsection{Simplicial and simple polytopes}
\label{sec:orgb33f178}
Polytopes with all proper faces being simplices.

Simple polytope: the dual is simplicial.

\(P\) is simple if and only if every vertex is contained in \(\dim P\) edges.
\subsubsection{Pyramid}
\label{sec:org481eafd}
Is convex hull of \(\{Q, p\} \subset \R^n\). Where \(\dim Q = n - 1, p \notin
    \aff Q\)

\(\vert P = \vert Q \cap \{p\}\)

\(f_k(P) = f_k(Q) + f_{k-1}(Q)\), \(0 \le k \le n-1\).

\(f_{-1}(Q) = 1 = f_{n-1}(Q)\), \(f_i(Q) = 0, i < -1, i > n-1\).
\subsubsection{Remarks}
\label{sec:org8f83514}
\begin{enumerate}
\item Any \(d\) polytope is \(0\) fold \(d\) pyramid.
\item A polytope \(P\) is an \(r\) fold, \$d\$-pyramid if it is a pyramid over an
\(r-1\) fold \(d-1\) pyramid \(Q\) and basis of \(P\) and \(Q\) are the same. The
base is a polytope of dimension \(d-r\).
\end{enumerate}
\subsubsection{Lemma}
\label{sec:org98acf25}
If \(Q_1 \approx Q_2\) and \(P_1, P_2\) are pyramids over \(Q_1\), \(Q_2\)
respectively, then \(P_1 \approx P_2\). Here \(\approx\) means combinatorially
isomorphic. \footnote{What is combinatorially isomorphic?}
\subsubsection{Lemma}
\label{sec:org985db90}
Let \(P\) be an \(r\) fold, \(d\) pyramid over \(Q\), then \$1\$-fold pyramid over
\(d-1\), \(G\).

\(f_k(P) = \sum_{i=0}^{r} \bionom{r}{1} f_{k-1}(Q)\)

\(f_k(P) = \sum_{i=0}^{k} \binom{i}{1}f_{k-i}(Q) = f_k(a) + f_{k-1}(Q)\).
\begin{enumerate}
\item Proof
\label{sec:orgd519325}
Induction hypothesis: Formula holds for \(r\) fold \(d\) pyramids (where \(d\) is
anything.)

\(P\) is an \(r+1\) pyramid over \(Q\).

\(P\) is a \(1\) pyramid over \(R\) where \(R\) is \(r\) pyramid over \(Q\).

\(f_k(P) = f_k(R) + f_{k-1}(R) = \sum_{i=0}^{r} \binom{r}{i} f_{k-1}(Q) + \sum_{i=0}^{r} \bionom{r}{i} f_{k-1-i}(Q)\)

This is equal to \(\sum_{i=0}^{r}\binom{r}{i} f_{k-1}(Q) +
     \sum_{j=1}^{r+1}\bionom{r}{j-1}f_{k-j}(Q)\).

\(f_k(Q) + \sum_{j=0}^{n}(\bionom{r}{j} + \bionm{r}{j-1}) f_{k-j}(Q) +
     f_{k-r-1}(Q)\)
\end{enumerate}
\subsubsection{Theorem}
\label{sec:org6b04fec}
If \(p\) is an \(r\) fold pyramid, then \(P^{*}\) is an \(r\) fold pyramid.
\begin{enumerate}
\item Proof
\label{sec:orgcbb1c58}
\(P = \conv{Q \cup \{p\}\}\), \(Q\) base, \(p\) is the appex of \(P\).

\(\hat{Q}\) is a vertex of \(P^{*}\) and \(p\) is a face of \(P^{*}\). These two
statements imply that \(q\) is a vertex of \(P^{*}\) and \(q \neq \hat{Q}\).
There is a facet of \(L\) of \(P\) such that \(q = \hat{L}\). \(Q\) is a base of
pyramid.

\(L\) is another facet. \(L\) is another facet of \(P\). This means that \(L\)
contains the appex. Now, duality, \(\hat{L} = q\) is contained in \(\hat{p}\),
which is a facet. Thus \(P^{*}\) is a pyramid.

What about \(r\) bigger than \(1\)? Do the induction.
\item Exercise
\label{sec:org16a2c2e}
What would be the Hasse diagram of a pyramid, given the Hasse diagram of
the base?
\end{enumerate}
\subsection{Bipyramid}
\label{sec:org653b898}
Let \(Q \subset R^n\) be an \(n-1\) polytope and \(I = [p, q]\) be a \(1\) dimensional
polytope such that, \(\relint Q \cap \relint I = \text{pint}\)

The polytope \(B = \conv{Q \cup \{p, q\}) = \conv(Q \cup I)\) is a bipyramid
over \(Q\).

Face of \(B\), \(F = B \cap H\), face of \(B\)

\begin{enumerate}
\item \(\ver  F \subset  \ver Q\)
\item \(p \in \ver F\)
\item \(q \in \ver F \implies \vert F - \{p\} \subset \vert Q\)
\item \(p, q \in \ver F\). This case is not possible.
\end{enumerate}
\subsubsection{Lemma}
\label{sec:orgc7afec9}
\(f_k(B) = f_k(Q) + 2f_{k-1}(Q)\), for \(0 \le k \le n-1\) Here \(k\) are
vertices. \(\dim B = n\). (Here \(f\) is sort of like the degree vector)
\begin{enumerate}
\item Proof
\label{sec:orgac2ece2}
\(\dim Q = n-1\).

\(f_{n-1}(Q) = 2f_{n-2}(Q)\).
\end{enumerate}
\subsubsection{\(r\) fold \(d\) bipyramid}
\label{sec:org6827f3b}
\(P\) is \(r\) fold \$d\$-bipyramid, if it a bipyramid over \(r-1\) fold \(d-1\)
bipyramid and the basis are the same.

What is \(d-1\) fold \(d\) bipyramid, \(d\) fold \(d\) bipyramid. This will in fact
be a cross-polytope.
\subsubsection{Lemma}
\label{sec:orgf48564d}
If \(Q_1 \approx Q_2\) and \(P_1\), \(P_2\) are bipyramids over \(Q_1, Q_2\)
respectively, then \(P_1 \approx P_2\).
\subsubsection{Cross Polytope}
\label{sec:orgcc773b7}
We're in \(\R^n\) and \(e_1, \cdots, e_n\) be a basis, then \(C^n = \conv\{e_1,
    \cdots, e_n, -e_1, \cdots, -e_n\}\). If \(\langle e_i, e_j \rangel =
    \delta_{ij}\), and \(\Vert e_i\Vert = \Vert e_j \Vert\) for all \(i, j\), then
\(C^d\) is regular cross polytope.\footnote{All polytopes in dimension 2 are either simple and simplicial.}

\(f_k(C^n) = 2^{k+1} \binom{n}{k+1}\).

How does one think about it? The typical \(k\) face of \(C^n\) is the convex
hull \(\{\epsilon_0, e_{i_0}, \cdots, e_{i_k}\}\). There was some argument
about counting.
\subsubsection{Ordrent}
\label{sec:orgaba7a38}
In \(n\) dimensional space, given a basis. Then instead of base, we look at
the orthogonal complements. These are hyperplane. These are called
\(\{e_1^\perp, e_2^\perp, \cdots, e_n^\perp\}\). Now, \(\R^n\), without these
\(n\) perpendicular things would have \(2^n\) components.
\subsection{Prism}
\label{sec:org0b2872d}
Let \(Q \subset \R^n\) be an \(n-1\) polytope and \(Q'\), its translate by a vector
in a corresponding linear subspace to the \(\aff Q\).

We use the definition that it is the convex hull of \(Q \cup Q'\).

What are the number of faces?

\(f_k(P) = 2f_k(Q) + f_{k-1}(Q)\), \(1 \le k\)

\(f_0(P) = 2f_0(Q)\).

What about \(n-1\)?
\subsubsection{\(r\) fold \(d\) prism}
\label{sec:org220e7e8}
A prism over \(r-1\) fold \(d-1\) prism. What is \(d-1\) fold \(d\) prism?

\(d-1\) fold \(d\) prism is the \(d\) fold \(d\) prism which is a \(d\) parallelotop.
\subsection{Lemma}
\label{sec:org09ffd06}
If \(Q_1 \approx Q_2\) and \(P_1, P_2\) prisms over \(Q_1\) and \(Q_2\), then \(Q_1
   \approx Q_2\).
\subsection{Cube}
\label{sec:org1a52292}
A \(d\) cube is \(d\) parallelotop or \([0,1]^d\).

\(f_k(\textup{cube}) = 2^{dk} \binom{d}{k}\) for \(0 \le k \le d\).
\end{document}