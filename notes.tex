% Created 2018-10-31 Wed 11:23
% Intended LaTeX compiler: pdflatex
\documentclass[11pt]{article}
\usepackage[utf8]{inputenc}
\usepackage[T1]{fontenc}
\usepackage{graphicx}
\usepackage{grffile}
\usepackage{longtable}
\usepackage{wrapfig}
\usepackage{rotating}
\usepackage[normalem]{ulem}
\usepackage{amsmath}
\usepackage{textcomp}
\usepackage{amssymb}
\usepackage{capt-of}
\usepackage{hyperref}
\usepackage[left=2cm, right=2cm, bottom=2cm, top=2cm]{geometry}
\usepackage{parskip}
\def\R{\mathbb{R}}
\def\Z{\mathbb{Z}}
\def\pos{\operatorname{pos}}
\def\relint{\operatorname{rel\ int}}
\def\conv{\operatorname{Conv}}
\usepackage[T1]{fontenc}
\author{Harikrishnan Mulackal}
\date{\today}
\title{Discrete Geometry 1}
\hypersetup{
 pdfauthor={Harikrishnan Mulackal},
 pdftitle={Discrete Geometry 1},
 pdfkeywords={},
 pdfsubject={},
 pdfcreator={Emacs 26.1 (Org mode 9.1.13)}, 
 pdflang={English}}
\begin{document}

\maketitle
\tableofcontents



\section{Lecture 2 \textit{<2018-10-17 Wed>}}
\label{sec:orge55c00a}

\subsection{Definitions}
\label{sec:org06efc22}
\subsubsection{Positive half}
\label{sec:orgf56b55a}
\subsubsection{Affine hyperplane}
\label{sec:org1fdf2fe}
\subsubsection{(Question) What is the space of all oriented hyperplanes of \(\R^n\)}
\label{sec:org868ffb3}
\subsubsection{Definition of pos?}
\label{sec:org281a276}
\subsection{Radon's theorem}
\label{sec:org1c44622}
Three cases

\begin{itemize}
\item If the size of M is greater than or equal to n+2, there is a radon partiiton
\item If the size of \(M\) is greater than or equal to \(n+1\), \(0\) is an appex of
\(M\) and \(0\neq M\). or \(|M| \ge n+2\).

then there is a partition of \(M_1\) and \(M_2\) of \(M\) such that \(\pos M_1
     \cap pos M_2 \neq \emptyset\).
\item partition is unique if and only i f
\begin{itemize}
\item \(|M| = n+2\) and \(n+1\) points of \(m\) are affinely dependent.
\item \(|M| = n+1\) and no \(n\) vectors of \(M\) are linearly independent.
\end{itemize}
\end{itemize}
\subsection{Definitions}
\label{sec:orgcb2f620}
\subsubsection{Radon partition}
\label{sec:orga1804bb}
A partition \(M_1, M_2\) such that \(\conv M_1 \cap conv M_2 \neq \emptyset\) is called a \textbf{Radon partition}.
\subsection{Proof}
\label{sec:org1eaf7ad}
\subsubsection{Part 1}
\label{sec:orgc223351}
Size of \(M\) greater than \(n+2\). Take \(x_1, \cdots, x_{n+2}\) pairwise disjoint points from \$M.

Then \(x_1, \cdots x_{n+2}\) are affinely dependent, meaning that that you can
choose scalars, real such that \(\sum \lambda_i x_i = 0\).

We can assume that for some \(1 \le j \le n+1\) holds

\(\lambda_1 >0, \cdots, \lambda_i >0\) and \(\lambda_{i+1} \le 0, \cdots, \lambda{n+2} \le 0\).

\(\lambda = \lambda_1 + \cdots + \lambda_j = -(\lambda_{i+1} + \cdots + \lambda_{n+2}\).

\(X=\frac1\lambda (\lambda_1 x_1 + \cdots \lambda_j x_j)\) a convex
combination \(\in \conv\{x_1, \cdots, x_j\}\).

$$\frac{1}{\lambda}(\lambda_1x_1 + \cdots + \lambda_{n+2}x_{n+2}) = 0$$


$$X= \frac{1}{\lambda}(\lambda_1x_1 + \cdots + \lambda_{j}x_{j})  = -\frac{1}{\lambda}(\lambda_jx_j + \cdots + \lambda_{n+2}x_{n+2}) = 0$$

a convex combination.

Thus \(X\) is inside both the intersection.
\subsubsection{Part 2}
\label{sec:org5600729}

\(\pos M\) is a cone with appex \ldots{} 

\textbf{A drawing}

For every point \(x_i \in M\), there is \(\alpha_i > 0\), such that \(\alpha_ix_i
    \in H'\),,

blah blah.

Keyword: matroids.
\subsubsection{Part 3}
\label{sec:org57c2176}
I didn't type.

(A part.) A partition is unique.

Let \(\vert M \vert = n+2\) and \(x_1, \cdots, x_{n+2}\) are affinely dependent implies
\(x_1, \cdots x_{n+1}\) (dropped one point) have to be on a affine hyperplane.
Then by Radon's theorem, there is a partition \(M_1'\) and \(M_2'\) such that
the convex hulls of these intersect. But then we can add the element
\(x_{n+2}\) to \(M_1'\) or \(M_2'\) to form two different partitions. Thus we have
two Radon partitions of \(M\) and thus this is a contradiction.

Case for \(|M| \ge n+3\). Let \(\tilde{M} \subset M\) and \(|\tilde{M}| = n+2\)
such that \(M-\tilde M \neq \emptyset\). Then \(\tilde M_1, \tilde M_2\) is a
partition of \(\tilde M\).

\(\conv \tilde M_1 \cap \conv \tilde M_2 \neq \emptyset\).

$$\tilde M_1 \cap (M-\tilde M), \tilde M_2$$

and $$\tilde M_2 \cap (M-\tilde M)$$ forms two different radon partitions and
again, we have a contradiction.
\subsection{Questions}
\label{sec:org6d2d10d}
\subsubsection{About affine maps}
\label{sec:org3e3b234}
We have an arbitrary affine map between a simplex (n+1 dimensional) and
\(\R^n\). implies, there exist faces of the simpleces such that the faces do
not intersect, but the images of the faces will intersect. Apparently this
follows from the Radon's theorem. A different formulation of Radon's theorem

\(a\colon T_{n+1} \rightarrow \R^n\)

\textbf{Question}: Replace the affine map by a continuous map and is it still true? \footnote{Wikipedia article about Radon'n theorem says that this is true. Crazy.}

\textbf{Question}: How many points in \(M \subset \R^d\) you should have to generate
that for \(n\ge 2\), there is a partition \(M_1, \cdots, M_r\) of \(M\) such that
the intersection of the convex hulls of \(M_i\) are non-empty. 

\textbf{Question}: More points, minimal number of points?
\section{Lecture 2 \textit{<2018-10-23 Tue>}}
\label{sec:orga418503}
\subsection{Review}
\label{sec:org0002a85}
\subsubsection{Radon's theorem}
\label{sec:org0889d26}
\begin{enumerate}
\item If \(M \subset \R^n\) and \(\vert M \vert \ge n+2\), then there exists a
partition \(M_1\) and \(M_2\) of \(M\) such that \(\conv M_1 \cap \conv M_2 \neq
       \emptyset\).
\item If \(M \subset \R^n\) and either \(\vert M \vert \ge n+2\) and \(0\neq M\) on
\(\vert M \vert \ge n+2\), then there is a partition of \(M_1\), \(M_2\) of \(M\)
such that \(\pos M_1 \cap \pos M_2 \neq \empty\)
\end{enumerate}
\subsection{Charatheodery's theorem}
\label{sec:org936b723}
\begin{enumerate}
\item Let \(M\subset \R^n\), then \(\conv M\) is the set of all convex combinations
of at most \(n+1\) points from \(M\). \footnote{What if \(M\) is a disc?}
\item Let \(M\subset \R^n\). Then \(\pos M\) is the set of all positive combinations
of at most \(n\) points from \(M\).
\end{enumerate}
\subsubsection{Proof}
\label{sec:org9ec3287}
\(x\in \conv M \implies\) there exists \(\lambda_1, \cdots, \lambda_n\) and \(x_1,\cdots, x_n \in M\). 

\(x=\lambda_1x_1+\cdots + \lambda_rx_r\) and \(\sum \lambda_i = 1\), \(\lambda_i \ge 0\).

Let the presentation be such that \(r\) is minimal. (We can do this because we
are taking minimum over natural numbers.) Let us assume that \(M \ge n+2\),
then there exists an affine dependence \(\mu_1x_2 + \cdots \mu_rx_r = 0\),
\(\mu_1 + \cdots + \mu_r = 0\) and not all \(\mu_i\) 's are zero.

(Basically the idea is that we assume the minimality of \(r\) and if \(r \ge
    n+2\), then there is an affine dependence, and then use this to contradict
the  minimality of \(r\).)\footnote{Solve the exercise for Helly's theorem.}
\subsubsection{Lemma about compactness of convex hull of compact set}
\label{sec:org6f6526b}
\(M^{n+1} \times \Delta \rightarrow M\). Here the space on the left is the set
of all \(n+1\) points of \(M\) and \(\Delta\) is a simplex.

Now, it follows from the fact that image of a compact set is compact.
\subsection{Nearest points map and supporting hyperplane}
\label{sec:org117affb}
\subsubsection{Lemma}
\label{sec:orgdab1ea7}
Let \(K\subset \R^n\) be closed and convex. Then for every \(x\in \R^n\), there
is unique point \(x^1 \in K\) such that $$\Vert x - x^1\Vert = \inf\Vert x -
    y\Vert = d(x, K)$$
\subsubsection{Proof}
\label{sec:org7491e98}
We can find a sequence of points \((y_n)\) in \(M\) such that the distance from
\(x\) is less than \(1/n\). Now, the sequence is Cauchy. Since, \(\R^n\) is
complete, it has to converge, and since \(K\) is closed, we are done. \footnote{Interestingly, convexity of the set is not used here. It's probably only
needed for the uniqueness.}

Uniqueness: Given \(X\), if there are two points \(x'\) and \(x''\) such that the
distances from \(x\) from these two points are the same. In the plane \(x, x',
    x''\), the triangle \(\Delta x x' x''\) exists. But then a perpendicular to the
side \(x'x''\) would be smaller than the distance to \(x'\) or \(x''\). This is a
contradiction. Hence the points have to be unique. (Here the convexity of
the set is used.)
\subsubsection{Definition of nearest point map}
\label{sec:orgf043580}
Given \(K\subset \R^n\) be a closed convex set. Then \(p_k \mathbb \R^n
    \rightarrow K\) is the nearest points map. (This is defined using the last
lemma.)

If \(x\in K\), then \(p_k(x) = x\). \(p_K\) is surjective. Usually it is not
injective, if \(K = \R^n\), then it is injective. 
\subsection{Properties of nearest point map}
\label{sec:org6260070}
\subsubsection{Definition (supporting hyperplane)}
\label{sec:org0b50c48}
A hyperplane \(H\) is a supporting hyperplane if \(a\) closed convex set in \(\R^n\) if 
$$H \cap K \neq \emptyset \textup{ and } K \subset H^- \textup{ or } K \subset H^+$$

If we take a \(u \in S(\R^n), \alpha \in \R^n\), \(H^+ = \{x\in\R^n \vert \langle x, u \rangle \ge \alpha\}\)
\(H^- = \{x\in\R^n \vert \langle x, u \rangle \le \alpha\}\).

A picture that I didn't draw 

Notions:
\begin{enumerate}
\item Supporting half space
\item Outer normal
\item Inner normal
\end{enumerate}
\subsubsection{Aim}
\label{sec:org13236e5}
We want to prove: Given a convex body and take a point in the boundary. I
want to prove that there is a supporting hyperplane (?)
\subsubsection{Lemma}
\label{sec:org3c15822}
Let \(\varphi \neq K \subset \R^n\) be a closed convex set. If \(x \in \R^n\setminus K\),
then the hyperplane \(H = \{y \in \R^n \vert \langle y, u \rangle = 1\}\) is a
supporting hyperplane of \(K\) at \(x' = p_k(x)\) where \(u=\frac{x-x'}{\langle
    x', x - x'\rangle}\).

A diagram I didn't draw (A convex body, x is a point outside, \(x'\) is the
closest element, meaning that \(x'\) is on the hyperplane and we have a
direction vector \(x - x'\), we normalize this vector. (\footnote{I think I made a mistake in what I wrote here.})) \footnote{If \(p_K\) is what we already know, we know that every point in \(p_k(\R^n \
K)\) \(\subset\) M\$ has a supporting hyperplane. We'll try to figure out more about
this set.}
\subsubsection{Proof}
\label{sec:org6e5a853}
\(H\) is a hyperplane and \(x' \in H\), then \(\langle x - x', x - x'\rangle \ge
    0 \implies \langle x, x-x' \rangle > \langle x', x-x'\rangle \implies \langle
    x, (x-x')/(\langle x', x-x'\rangle) \implies x\in H^+\)

Now we assume that \(H\) is not a supporting hyperplane, which means that
there is as point \(y\) inside \(K\cap (H+\setminus H)\). Consider the
triangle \(\Delta x x' y\). Since \(x\) is perpendicular to the \(yx'\), the angle
\(yx_1 x\) is actute. We kinda want to prove that there is a point on the line
segment that would minimize the distance from \(x\). The argument is similar
to the argument for last theorem. (The perpendicular from \(x\) would give a
point on the segment \(x'y\) that would be the minimum.) \footnote{I think I made mistakes in framing at the beginning of the paragraph.}
\subsubsection{Lemma}
\label{sec:orgd9e06d6}
Let \(K\subset \R^n\) be a closed convex set and \(x\in \R^n \setminus K\).
For a point \(y\) on the half-line emanating from \(x'=p_k(x)\) and containing
\(x\) holds

\(y' = p_K(y)=p_K(x) = x'\)
\subsubsection{Proof}
\label{sec:org2428fb3}
Let \(y \in [x', x]\), assume that \(y' \neq x'\). We'll try to arrive at a contradiction.

\(\Vert x - x' \Vert = \Vert x - y \Vert + \Vert y - x'\Vert \ge \Vert x -
    y\Vert + \Vert y - y'\Vert\) (The second part follows from the fact that \(y'\)
is the point in \(K\) that is closest to \(y\).)

We apply the inequality of triangle we get that \(\Vert x - x' \Vert \ge
    \Vert x - y\Vert\). This is a contradiction.

We do something similar when \(x\) is an element in the line segment \([y,
    x']\). (Not exactly similar, but try to arrive at a contradiction from
drawing some triangles and what-not.)
\subsubsection{Lemma Busemann and Faller's lemma}
\label{sec:org4fff481}
The function \(p_K\) does not increase the distance, therefore it is Lipschitz
with constant \(1\) and is uniformly continuous. This means that \(\Vert
    p_k(x) - p_K(y) \Vert \le \Vert x - y\Vert\).
\subsubsection{Proof}
\label{sec:orgc8d778e}
We assume that \(x' = p_K(x) \neq y' = p_K(y)\). (We draw a diagram.)

We kinda use principles similar to the last two theorems. I skipped writing
the proof.
\section{Lecture 3 \textit{<2018-10-24 Wed>}}
\label{sec:org1ff1fe4}
\subsection{Review}
\label{sec:orgfaf5293}
\subsubsection{Nearest point map}
\label{sec:orgf3cdfa4}
The definition of the nearest point map for a convex set.

Recall that we use completeness of Real numbers for the existence of the map. \footnote{Did I write this statement correctly?}
\subsubsection{Some properties}
\label{sec:orgbc0b31d}
The nearest point map is identity in \(K\). 

Every point \(y\) on the half line emanating from \(x'\) containing is in the
fiber of \(x'\) with respect to \(p_K\).

\(f_K\) is a Lipschitz function with constant \(1\) and is hence continuous. 

Supporting hyperplane \(H \colon H \cap K = \emptyset\), \(K\subset H^{-}\). 
\subsubsection{Lemma}
\label{sec:orga808d71}
If \(x \in \R^n - K\), then \(H=\{y \colon \langle y, x-x'\rangle = \langle x',
    x-x'\rangle \}\) is a supporting hyperplane of \(K\) at \(x'\).

The lemma says that at every point outside of \(K\), we can find a supporting
hyperplane. What we need to prove is that at every point on the boundary we
can find a supporting hyperplane. 
\subsection{Theorem}
\label{sec:org70657b3}
Let \(K\subset \R^n\) (here \(K\) is not equal to \(\R^n\) be closed convex set.
Then \(K\) is equal to the intersection of all its supporting half-spaces. 
\subsubsection{Proof}
\label{sec:orgc683e97}
Because \(K\) is not \(\R^n\), we have a point in the difference. Then there is
at least one supporting hyperplane, and therefore a supporting half space.
Let \(K'\) be the intersection of all of it's supporting hyperplanes of \(K\).
It is clear that \(K\) is a subset of \(K'\). To prove the inclusion from the
other side:

Let \(k'\) be an element in \(K'\). Then there exists a supporting hyperplane
\(H\) at \(x'=f_K(x)\) such that \(K \subset H^{-}\) and \(x \in inf H^{+}\). Thus
\(H\) separates \(H\) and \(K\), and more importantly, \(x\) is not an element of
\(K'\).\footnote{We're interested in spaces that can be formed by finitely many
intersections of hyperplanes. These will be called Polyhedra. An non-example is
a disc.}
\subsection{Theorem}
\label{sec:org18f285e}
Let \(K\subset \R^n\) a closed convex set and \(x\in \partial K\). Then there
exists a supporting hyperplane for \(K\) containing \(x\). 
\subsubsection{Proof}
\label{sec:org101f826}
We define the boundary of \(K\) first. Let \(x\in \partial K \iff (\forall U
    \in x \textup{ and open }) U \cap K \neq \emptyset\) and \(U\cap K^{c} \neq
    \emptyset\) and \(x_0 \in K\).

If \(x_0\) is a point in the boundary of \(K\), then there is a sequence \(y_n
    \in \R^n\) such that \(x_0\) is the limit of \(y_n\).

For every point \(x_n = f_K(y_n)\), there is a supporting hyperplane \(H_n\) at
\(x_n\). Let \(s_n\) be a sequence of half lines emanating from \(x_n\)
perpendicular to \(H_n\). Let \(S\) be a sphere with center at \(x_0 \in H\) of
small radius. Then this half line will intersect \(S\) at one point. Notice
that \(y_n\) is also an element of \(S_n\), then 

\(x_0 = \lim f_K(y_n')\) and \(y'_{k_n}\) subsequence of \(y_n'\) converging in
\(S, y_{k_n}' \rightarrow y_0 \in S\) and \(x_0 = \lim f_k(y_n') = \lim
    f_K(y_{k_n}')\) and \(y_0 = lim y_k' \implies f_k(y_0) = \lim f_k(y_{k_n}')\)
and \(x_0 = f_k(y_0)\) and \(y_0 \neq x_0\). \footnote{I don't understand what's happening at the end}
\subsection{Faces and Normal Cone}
\label{sec:org60e3bff}

\subsubsection{Definition}
\label{sec:org7e0ea16}
Let \(K\subset \R^n\) be a closed convex set. A face \(F\) of \(K\) is a subset of
\(K\) is a subset of the form \(F = K \cap H\) where \(H\) is some supporting
hyperplane of \(K\). 

Such a face is called a proper face while \(\phi\) and \(K\) are also faces but
called non-proper. (A diagram with \(\emptyset\) and \(K\).)

Examples: Triangles (here faces are the edges.) For a disc, then the faces
are points on the boundary. For a cube, the faces are the faces of the cube.
\subsubsection{Lemma about convexity of face}
\label{sec:org686621e}
Every proper face of \(K\) is a closed convex set.
\subsubsection{Dimension}
\label{sec:orgd36c8a2}
If \(F\) is a face of \(K\) and \(m=\dim F\), (Let affine hull of \(K\) is \(\R^n\).)
\begin{enumerate}
\item then \(m=0\) we call \(F\) a vertex of \(K\)
\item If \(m=1\) we call \(F\) an edge of \(K\)
\item If \(m=n-1\) we call \(F\) a \textbf{facet} of \(K\)
\item If \(m=n-2\) we call \(F\) a \textbf{ridge} of \(K\).
\end{enumerate}
\subsection{Lemma}
\label{sec:org16d6816}
Let \(F_0\) be subset of \(F_1\) faces of \(K\), then \(F_0\) is a face of \(F_1\).
\subsubsection{Proof}
\label{sec:org8507e35}
\(F_0\) is a face of \(K \implies\), therefore \(F_0 = K \cap H\), where \(H\) is a
supporting hyperplane for \(K\). \(H\) supporting hyperplane for \(K\) and
therefore for \(F_1\).

\(F_1 \cap H \subset K \cap H \subset F_1 \cap H\)

\(F_0 = F_1 \cap H\). 
\subsubsection{Remark}
\label{sec:orgbd1c829}
The converse of the lemma does not hold. \(F_2\) is a face of \(K\) and \(F_0\) is
a face of \(F_2\) implies \(F_2\) is a face of \(K\). The last statement is not
true. Notice that in the above proof we need both of them to be faces of
\(A\).

The picture: A square with a half-disc glued to the right. \(F_0\) be a vertex
on the right side and \(F_1\) be the edge of the square containing \(F_0\). \footnote{Apparently the statement would be true for polytopes, i.e., the converse
holds for polytopes. This is one of the reason we're interested in polytopes.}
\subsection{Lemma}
\label{sec:orgeb61b7c}
Let \(F_1, \cdots, F_k\) be faces of \(K\), then \(F=F_1\cap \cdots \cap F_k\) if a
face of \(K\).
\subsubsection{Proof}
\label{sec:org359fa3b}
\(F_i = K \cap H_i\), where \(H_i = \{y \vert \langle y, u_i \rangle = 0\}\).
(\(K \subset H_i^{-1}\) \footnote{We need the next statement for making this assertion.}) We can assume that \(0 \in F\) which is the
intersection of all of them \(0 \in F = F_1 \cap \cdots F_k\).

\(u=u_1+\cdots+u_k\) (we can assume without loss of generality that \(u \neq
    0\); this can be attained by scaling one or more \(u_i\).)

\(H=\{y\vert \langle y, u\rangle = 0\}\). will be a supporting hyperplane for
\(K\) and \(F = K \cap H\). \(K\subset H^-\), \(y\in K\), \(\langle u, u\rangle =
    \langle y, u_1\rangle + \cdots + \langle y, u_k\rangle \le 0 + \cdots + 0\). \footnote{This is why we assumed that \(0\) is in \(K\), otherwise we'll have to play
around.}

The last statement implies that \(y\)

\(F = K \cap H\).

\(y\in F = F_1 \cap \cdots \cap F_k = (K\cap H_1) \cap \cdots \cap (K\cap H_k)\)

\(\langle y, u_1 \rangle = 0, \cdots, \langle y, u_k\rangle = 0\).

\(\langle y, u_1 + \cdots + u_k \rangle = 0\).

\(\langle y, u \rangle = 0 \implies y \in H\). 

\(y \in F \cap H \subset K \subset H\). 

\(y \in K \cap H \implies y \in K\) and \(y\in H\). 

\(\langle y, u_i \rangle \le 0\) 

\(\langle y, u \rangle = 0 = \langle y, u_1 \rangle + \cdots + \langle y,
    u_k\rangle\). \footnote{Question: what about infinite intersection.}
\subsection{Lemma}
\label{sec:org6e9efee}
\begin{itemize}
\item Let \(F\) be a face of a closed convex set \(K\) and \(x, \tilde x\) be an element
\end{itemize}
of the relative interior of \(F\). Then any supporting hyperplane of \(K\)
containing \(x\) must contain \(\tilde x\).
\begin{itemize}
\item If \(F, F'\) are faces of \(K\) and \(\relint F \cap \relint F \neq \emptyset\),
ten \(F = F'\).
\end{itemize}
\subsubsection{Proof}
\label{sec:org36b7dc5}
\(H\) supporting for \(F\). I didn't write this. \footnote{This means that we can choose a supporting hyperplane by choosing a
point inside relative interior.}
\section{Lecture 4 \textit{<2018-10-30 Tue>}}
\label{sec:orgb64c389}
\subsection{Review}
\label{sec:org4284d81}
He did a review of stuff. 

\begin{enumerate}
\item \(K\) closed convex set and \(H\) is a supporting hyperplane of \(K\). Meaning
that \(F = K \cap H\) is a face. \(\phi_1, K\) (improper) face.
\item \(F\) face of \(K\) \(\implies F\) closed and convex.
\item \(F_1 \subset F_1\) faces of \(K \implies F_0\) a face of \(F_1\). Whereas the
converse of the statement is not true. \footnote{Remember the example with a square and a half disc glued to the right side of the square?}
\end{enumerate}
\subsection{Definition (Normal cone)}
\label{sec:org6ab5761}
Let \(K\subset \R^n\) be a closed convex set and \(x\in K\). The \textbf{Normal cone} at
\(x\) is the set at \(x\) is the set $$N(x) = -x + p_K^{-1}(\{x\})$$

The normal cone at \(x\) always contains \(0\). We'll draw some examples.

\begin{enumerate}
\item \textbf{A closed convex interval in \(\R\)}. Take a point \(x\) inside the interval.
Then \(N(x) = 0\). This is because the set of all points such that the
closest point is \(x\) is just \(x\).
\item If we go at the boundary of the convex set, then the set of points that
are closest to the point is the point and the whole half line containing
the point. Now \(N(x)\) is \([0, \infty)\) after translation. We can make a
similar argument for the point on the other side of the boundary.
\item \textbf{An interval in the plane}: let's say \([1, 3]\) inside \(\R^2\). Now, for
\(2\), there is a perpendicular line that is closest to \(2\). Now, if we
translate it, we get a line perpendicular to \(0\). Whereas, for \(3\) and
\(0\), they would be two dimensional spaces (half spaces.) We can get one
form another by doing orthogonal complement.
\item \textbf{A triangle inside plane}. All the points inside would give us \(0\).
Whereas, for a point on one of the edge (other than vertex), \(N(x)\) would
be a line perpendicular to the edge. For an edge, it would be a
two-dimensional space.
\item \textbf{Remark}: Notice that for all these examples, we were able to partition
the entire space using \(N(x)\). (I think the partition thing we are talking
about is about \(p^{-1}_K\). \(N(x)\) would always contain \(0\).
\end{enumerate}
\subsection{Lemma}
\label{sec:orgc831fad}
\(N(x)\) is a closed convex cone. It consists of \(0\) and all outer normals of
\(K\) in \(x\). If \(x \in \int K\), then \(N(x) = \{0\}\). 
\subsubsection{Proof}
\label{sec:org652ee47}
\(\lambda \ge 0, u \in N(x) \implies \lambda x \in N(x)\)

\(u, v \in N(x) \implies u + v \in N(x)\)

Without loss of generality, we can assume that \(x=0\). 

\(u\in N(0) \implies u \in p^{-1}_K\) and a lemma gives us that \(\lambda v\in
    p^{-1}_K(0)\) implies that \(\lambda u \in N(0)\).

\(u, v \in N(0) \implies 0 = p_K(u) = p_K(v)\), \(H_u = \{x\vert \langle u, x
    \rangle = 0\}\)

\(K\subset H_u^{-1}\). Supporting hyperplane at \(0\) of \(K\). \(H_u = \{x \vert
    \langle v, x\rangle = 0\}\), and \(K\subset K_v^{-1}\).

\(x \in K, \langle u + v, x\rangle = \langle u, x\rangle + \langle v,
    x\rangle \le 0 + 0 \le 0\).

\(x \in H^{-1} \implies K \subset H^{-1}\).

\(H\) is a supporting hyperplane, then \(p_{K}(u+v) = 0\).

What we proved is that, if we take a point, the positive multiple is inside.
We also proved that if there are two points inside, then the sum of them is
also inside. So it's like a cone. What about closed?

\(N(x) = -x + p^{-1}(\{x\}\). Now, because \(\{x\}\), is closed and \(p_K\) is
continuous, then inverse image is closed. Because the translation is an
isometry, we are done.
\subsection{Definition (Dual cone)}
\label{sec:org082ac81}
Let \(G\) be a cone, then \(\sigma = \{u \vert \langle \sigma, u\rangle \ge 0 \}\)
is called the \textbf{dual cone}. 
\subsection{Lemma}
\label{sec:org15f4260}
If \(\sigma\) is a cone with appex \(0\), then \(N_\sigma(0) = -\sigma\).\footnote{He didn't prove this, but it's obvious.}
\subsection{Lemma}
\label{sec:org05e5700}
Let \(F\) be a face of a closed convex set of \(K\) and \(x, \tilde{x} \in \relint
   F\), then \(N(x) = N(\tilde x)\).
\subsubsection{{\bfseries\sffamily TODO} Proof}
\label{sec:org229517e}
The idea is that if there are two points in the relative interior of a face,
then the supporting hyperplane for these points are the same. We look at all
the normals at \(x\) and \(\tilde x\). \footnote{I think I'm missing some stuff.}
\subsection{A random story}
\label{sec:org57508ca}
\(P \rightarrow \{F \colon F \textup{ a face of P}\), for every face, we can
talk about \(N(F)\) instead of a point in the relative interior. These two
sets, we put inclusion as a relation, these are anti-isomorphic \footnote{I think it means, the inclusion becomes opposite in the other space.} These
have some group structure and later can be used to construct affine Toric
variety. \footnote{The construction is similar to the construction of a toric variety.}
\subsection{Definition}
\label{sec:org599b005}
If \(F\) is a face of a closed convex set \(K\) and \(x\in \relint F\), then \$N(x)
is denoted by \(N(F)\) and is called the cone of normals of \(K\) in \(F\). \footnote{Connection with analysis: For a smooth real valued function from \(\R\),
we have a unique normal, whereas for a non-smooth point, there are several
different normals (or supporting hyperplane.) We'll do something similar in our
course.}
\subsection{Theorem}
\label{sec:org7b05145}
Let \(K\) be a convex body in \(\R^n\) and \(x(F)\) are of the relative interior
points in \(F \neq \emptyset\) or \(K\). Then \(\{\relint N(x(F)) \vert F \text{
   face of } K\} = \{ \relint N(F) \vert F\textup{ face of } K \}\) is a
partition of \(\R^n\).
\subsubsection{{\bfseries\sffamily TODO} Proof}
\label{sec:orgc862deb}
Since \(K\) is bounded, there exists \(\alpha\) non-negative, such that \(K\) is a
subset of \(H^{-1}(u, \alpha)\) where $$H(u, \alpha) = \{x \vert \langle x, u
    \rangle =\alpha\}$$

Let's take the intersection \(\cap_{K \subset H^{-1}}(u, \alpha) H^{-1}
    H^{-1}(u, \alpha)\)

There was a nice diagram. 
\subsubsection{Random stuff}
\label{sec:org5310ee8}
\(\forall u \in \R^n - \{0\}\), there exists a face \(F\) of \(K\), \(u\in N(F)\)
and \(0 \in N(K)\).

\(x \in \relint F\), and \(u\) is an outer normal of \(x\), then \(u\in \relint
    N(F)\).\footnote{This may or may not be true. Wasn't discussed in the class.}

He did an example with tetrahedra.

\(u \in \relint N(F_1) \cap \relint N(F_2)\), \(u\ in \relint x(F_1) \cap
    \relint x(F_2)\). This means that if we take \(u\) and add a point \(x(F_1)\),
\(p_K(u + x(F_1)) = x(F_1)\). This means that \(p_K(u + x(F_2)) = x(F_2)\)

(I missed parts of this argument.) We used boundedness of the convex body.
If it is unbounded, the family of normal cones do not cover.
\subsection{Definition (Normal fan)}
\label{sec:orgd9a6a84}
The family \(N(F)\) of \(K\) is called the normal fan of \(K\)
\subsection{Support and distance function}
\label{sec:org175d2fd}
\subsubsection{Definition (support function)}
\label{sec:orgd9ba5aa}
Let \(K\) be a non-empty convex body. The function \(h_K\) that maps \(\R^n
    \rightarrow \R\), \(h_K(x) = \sup_{x\in K} \langle u, x\rangle\) is the
\textbf{support function} of \(K\).\footnote{We use the fact that every compact function has a supremum.}

We can also say it is the supremum over a fixed \(x_0\).

There was a diagram
\subsubsection{Question?}
\label{sec:orgd9bbf68}
Given a ball, what is the normal fan of the space?

If we take an interior point, then we have \(0\). So we should go to the
boundary. But each point has a supporting hyperplane. Which means, that the
normal fan is all the half lines. The normal fan is a sphere. The normal fan
can be ugly when we have smooth convex body. But for polytopes, it's much
nicer.
\end{document}